% --- To be compiled with XeLaTeX ---
% ---      Encoding: UTF-8        ---

\documentclass[a4paper, oneside, 11pt]{article}

%fontspec package provides a configurable interface for font selection, and allows complex font choices to be named and later reused. It's needed for XeLaTeX
\usepackage[cm-default]{fontspec}

% Unicode support
\usepackage{xunicode}
\usepackage{xltxtra}

% Default words and phrases in Greek (e.g. 'Περίληψη' instead of 'Abstract'). Also contains hyphenation rules for Greek Language
\usepackage{xgreek}

% Mathematical fonts, theorems etc.
\usepackage{amsfonts}
\usepackage{amsmath}
\usepackage{amsthm}

% Default page layout for consuming a larger portion of the page.
\usepackage{fullpage}

% Greek fonts (Computer Modern)
\setmainfont[Mapping=tex-text]{CMU Serif}

% Auxiliary commands
\newcommand{\HRule}{\rule{\linewidth}{0.5mm}}

\usepackage{tikz}

\newtheorem{thm}{Θεώρημα}
\newtheorem{lm}[thm]{Λήμμα}

\theoremstyle{definition}
\newtheorem{defn}[thm]{Ορισμός}

\begin{document}

\begin{titlepage}
\begin{center}

\includegraphics[width=0.3\textwidth]{./pyrforos.png}
\includegraphics[width=0.2\textwidth]{./uoa.png}\\[1cm]

\textsc{\LARGE Σχολή Ηλεκτρολόγων Μηχανικών και Μηχανικών Υπολογιστών}\\[1.5cm]

\HRule \\[0.4cm]
{\huge \bfseries Γραφοθεωρία\\
\LARGE Ομάδα Ασκήσεων No. 2}\\[0.4cm]

\HRule \\[1.5cm]

\begin{center}
\textbf{Ομάδα 7}\\
Αξιώτης Κυριάκος\\
Αρσένης Γεράσιμος
\end{center}

\vfill

{\large \today}
\end{center}

\end{titlepage}


\section{Βαθμοί, κύκλοι, μονοπάτια}

\begin{enumerate}
\item[1.7 ($\star$)]
   Δείξτε ότι αν για κάποιο γράφημα $G$ ισχύει ότι
   $\textbf{διάμετρος}(G) \geq 2$, τότε το ίδιο θα ισχύει και για
   το απόκεντρο του G.

   \begin{proof}
   Έστω $\textbf{αποκ}(G)$ το σύνολο των κορυφών του $G$ που ανήκουν στο
   απόκεντρο και $H = G_{\textbf{αποκ}(G)}$ το εναγόμενο από το απόκεντρο
   υπογράφημα.

   Έστω ότι ήταν $\textbf{διάμετρος}(H) \leq 1$. Επειδή το απόκεντρο
   πρέπει να περιέχει τουλάχιστον 2 κορυφές, η διάμετρος δεν μπορεί
   να είναι 0 άρα έχουμε ότι
   $\textbf{διάμετρος}(H) = 1$, δηλαδή το $H$ είναι πλήρες γράφημα.

   Τότε όμως η διάμετρος του $G$ δεν μπορεί να είναι μεγαλύτερη ή ίση
   του 2 γιατί όλες οι αντιδιαμετρικές κορυφές βρίσκονται στο
   απόκεντρο και το απόκεντρο είναι πλήρες.

   Συνεπώς αν $\textbf{διάμετρος}(G) \geq 2$ τότε και
   $\textbf{διάμετρος}(H) \geq 2$.
   \end{proof}

\item[1.8 ($\star$)]
   Προσδιορίστε τη μέση απόσταση δύο κορυφών του γραφήματος $Q_r$ (δηλ.
   το μέσο όρο των αποστάσεων για όλα τα δυνατά ζεύγη διακεκριμένων κορυφών).

   \begin{proof}
      Ως γνωστόν οι κορυφές του υπερκύβου μπορούν να αριθμηθούν
      με δυαδικές συμβολοσειρές μήκους $r$. Δύο κορυφές συνδέονται
      με ακμή ανν οι συμβολοσειρές τους διαφέρουν μόνο σε μία θέση.

      Έστω μία κορυφή $x$. Το πλήθος των κορυφών που βρίσκονται σε
      απόσταση $d$ είναι ίσο με το πλήθος των κορυφών που οι συμβολοσειρές τους
      διαφέρουν σε $d$ ακριβώς θέσεις σε σχέση με την $x$. Δηλαδή υπάρχουν
      ${r \choose d}$ κορυφές σε απόσταση $d$.

      Συνεπώς έχουμε:

      \begin{align*}
         E[d] &= \frac{1}{{n(G) \choose 2}} \sum_{u, v \in V(G): u \neq v}
                  d(u, v)\\
              &= \frac{1}{{2^r \choose 2}} \sum_{u \in V(G)}
                  \sum_{v \in V(G): v \neq u} d(u, v)\\
              &= \frac{1}{\frac{2^r \cdot (2^r-1)}{2}}
                  \sum_{u \in V(G)} \sum_{k=1}^{r} k \cdot {r \choose k}\\
              &= \frac{2}{2^r (2^r - 1)} n(G) \sum_{k=1}^{r}
                  r {r-1 \choose k-1}\\
              &= \frac{2 \cdot 2^r \cdot r}{2^r (2^r - 1)} \sum_{k=0}^{r-1}
                  {r-1 \choose k}\\
              &= \frac{2 \cdot r \cdot 2^{r-1}}{2^r - 1}\\
              &= \frac{r \cdot 2^r}{2^r - 1}
      \end{align*}

   \end{proof}

\item[1.9 ($\star$)]
   Για κάθε θετικό ακέραιο $\alpha$ και για κάθε γράφημα $G$, το $V(G)$
   περιέχει περισσότερες από $\left( 1-\frac1\alpha \right) \cdot n(G)$
   κορυφές βαθμού αυστηρά μικρότερου του $2\alpha\delta^*(G)$.

   \begin{proof}
      Σύμφωνα με τον ορισμό έχουμε ότι
      $\delta^*(G) = \max \{ k\ |\ \exists H \subseteq G \text{ με }
       \delta(H) \geq k \}$.

      Θέλουμε να δείξουμε ότι:

      \[
         | \{ u\ |\ u \in V(G) \land d(u) < 2\alpha\delta^*(G) \} |
            > \left( 1 - \frac1\alpha \right) n(G)\\
      \]

      Έστω λοιπόν, προς απαγωγή σε άτοπο ότι:
      \begin{align*}
         | \{ u\ |\ u \in V(G) \land d(u) < 2\alpha\delta^*(G) \} |
            &\leq \left( 1 - \frac1\alpha \right) n(G)\\
         \Leftrightarrow
         n(G) - | \{ u\ |\ u \in V(G) \land d(u) \geq 2\alpha\delta^*(G) \} |
            &\leq \left( 1 - \frac1\alpha \right) n(G)\\ 
         \Leftrightarrow
         | \{ u\ |\ u \in V(G) \land d(u) \geq 2\alpha\delta^*(G) \} |
            &\geq \frac1\alpha n(G)\\
      \end{align*}

      Ισχύει όμως:

      \begin{equation}
         \label{eq1.9.1}
         2m(G) = \sum_{u \in V(G)} d(u) \geq
         \sum_{u \in V(G): d(u) \geq 2\alpha\delta^*(G)} d(u)
         \geq \frac1\alpha n(G) \cdot 2\alpha\delta^*(G)
         = 2n(G)\delta^*(G)
      \end{equation}

      \[ m(G) \geq n(G) \cdot \delta^*(G) \Leftrightarrow
         \epsilon(G) \geq \delta^*(G) \]

      Από το Πόρισμα 3.1 των σημειώσεων του μαθήματος γνωρίζουμε ότι
      $\delta^*(G) \geq \max \{ \epsilon(G), \delta(G) \}$
      συνεπώς θα έχουμε:

      \[ \epsilon(G) = \delta^*(G) \]

      Άρα οι ανισότητες στη σχέση \ref{eq1.9.1} θα πρέπει να είναι ισχυρές,
      δηλαδή:

      \[ 
         \sum_{u \in V(G)} d(u) =
         \sum_{u \in V(G): d(u) \geq 2\alpha\delta^*(G)} d(u)
         = \frac1\alpha n(G) \cdot 2\alpha\delta^*(G)
      \]

      Δηλαδή οι μόνες κορυφές με $d(u) < 2\alpha \delta^*(G)$
      θα πρέπει να είναι απομονωμένες και επιπλέον οι υπόλοιπες κορυφές
      να έχουν βαθμό ακριβώς $d(u) = 2\alpha\delta^*(G)$ και να είναι
      ακριβώς $\frac1\alpha n(G)$ σε πλήθος. Τότε όμως, αν αφαιρέσουμε
      τις απομονωμένες κορυφές θα μείνει ένα $(2\alpha\delta^*(G))$-κανονικό
      γράφημα και έτσι $\delta^*(G) \geq 2 \alpha \delta^*(G)
      \Leftrightarrow \alpha \leq \frac12$.

      Αυτό όμως είναι άτοπο γιατί υποθέσαμε ότι ένα σύνολο κορυφών
      έχει πληθάριθμο $\leq \left( 1 - \frac1\alpha \right) n(G) < 0$.

   \end{proof}

\item[1.10 ($\star\star$)]
   Κάθε γράφημα $G$ με τουλάχιστον 2 κορυφές και
   $\epsilon(G) \geq 2$, έχει περιφέρεια το πολύ $2 \cdot \log_2 (n)$.

   \begin{proof}
Έστω το μικρότερο σε πλήθος κορυφών γράφημα, το οποίο έχει τουλάχιστον 2 κορυφές και $\epsilon(G) \geq 2$ και ο ελάχιστός του κύκλος είναι μεγαλύτερος από $2\cdot \log_2 (n)$. Αν αυτό το γράφημα περιέχει
μια κορυφή βαθμού 1, τότε αφαιρώντας την οι ακμές μειώνονται κατά 1 και οι κορυφές μειώνονται κατά 1, άρα $\epsilon (G') = \frac{m'}{n'}=\frac{m-1}{n-1}\geq \frac{2n-1}{n-1}\geq 2$. (Επίσης το υπόλοιπο
γράφημα έχει τουλάχιστον μια ακμή, αφού αν πριν τη διαγραφή είχε μόνο μία, θα είχαμε πυκνότητα το πολύ $1/2$). Αυτό όμως είναι άτοπο λόγω της υπόθεσης ελαχιστότητας. Αντίστοιχα, αν περιέχεται μια κορυφή
βαθμού 2, τότε αφαιρώντας την έχουμε $m'=m-2$ και $n'=n-1$, άρα $\epsilon(G')=\frac{m'}{n'}=\frac{m-2}{n-1}\geq \frac{2n-2}{n-1}=2$, το οποίο είναι και πάλι άτοπο για τον ίδιό λόγο με παραπάνω. Άρα όλες
οι κορυφές του γραφήματος έχουν βαθμό τουλάχιστον 3. Έστω τυχαία κορυφή $v$ και η αποσύνθεση απόστασης από την $v$ είναι τα σύνολα $V_0$, $V_1$, $V_2$, ..., $V_k$, όπου $V_i$ το σύνολο κορυφών που 
βρίσκονται σε
απόσταση $i$ από την $v$. Γνωρίζουμε ότι υπάρχουν ακμές μόνο στο εσωτερικό
ενός συνόλου ή μεταξύ διαδοχικών συνόλων της αποσύνθεσης. Αν μια κορυφή στο επίπεδο $i$ έχει δύο ακμές προς το επίπεδο $i-1$, τότε σχηματίζεται
κύκλος με μήκος το πολύ $2\cdot i$. Για να μην καταλήξουμε σε άτοπο, θα πρέπει $i>\log_2 (n)$. Επίσης, αν μια κορυφή στο επίπεδο $i$ έχει ακμή προς κάποια άλλη κορυφή στο ίδιο επίπεδο, τότε σχηματίζεται
κύκλος με μήκος το πολύ $2\cdot i + 1$. Για να μην καταλήξουμε σε άτοπο, θα πρέπει $i>\log_2 (n) - \frac{1}{2}$. 
Από αυτά συμπεραίνουμε ότι οι κορυφές στα επίπεδα $i \leq \log_2 (n) - \frac{1}{2}$ δεν έχουν ακμές προς το ίδιο επίπεδο, και έχουν το πολύ μία ακμή προς το πάνω επίπεδο. 
Άρα έχουν (η κάθε μία) τουλάχιστον 2 ακμές προς το κάτω επίπεδο. Μάλιστα, αυτές οι ακμές είναι προς διαφορετικούς κόμβους, αφού όπως είπαμε
παραπάνω αν μια κορυφή έχει δύο ακμές προς τα πάνω θα πρέπει να είναι σε επίπεδο με $i>\log_2 (n)$. Στο επίπεδο $i$, με $i\leq\log_2 (n)$ έχουμε λοιπόν τουλάχιστον $3\cdot 2^{i-1}$ κορυφές. Επειδή συνολικά
έχουμε $n$ κορυφές, θα πρέπει $3\cdot 2^{i-1} \leq n$, δηλαδή $i \leq \log_2 (\frac{n}{3}) + 1 = \log_2 (n) - \log_2 (3) + 1 \leq \log_2 (n) - \frac{1}{2}$. Συνεπώς το τελευταίο επίπεδο δεν μπορεί να
απέχει περισσότερο από $\log_2 (n) - \frac{1}{2}$. Οι κορυφές αυτού του επιπέδου, όμως, όπως είπαμε παραπάνω, δεν μπορούν να έχουν ακμές προς το ίδιο επίπεδο, και το πολύ μία ακμή προς το παραπάνω επίπεδο.
Αυτό είναι άτοπο γιατί έχουν βαθμό τουλάχιστον 3. Άρα η περιφέρεια κάθε γραφήματος με τουλάχιστον 2 κορυφές και πυκνότητα τουλάχιστον 2 είναι το πολύ $\log_2 (n)$.
   \end{proof}
\end{enumerate}

\section{Άκυκλα γραφήματα}

\begin{enumerate}
   \item[2.9 $(\star)$]
      Έστω $G = T_1 \cup T_2$ όπου $T_1$ και $T_2$ είναι δέντρα. Δείξτε
      ότι $\exists c \in \mathbb{N} : \delta^*(G) \leq c$ και βρέστε
      την μικρότερη σταθερά $c$ για την οποία $\delta^*(G) \leq c$ για κάθε
      γράφημα που είναι ένωση δύο δέντρων

      \begin{proof}
         Αρχικά θα δείξουμε ότι $c \leq 3$. Έστω ότι ήταν $c \geq 4$
         δηλαδή έστω ότι υπήρχαν δέντρα $T_1, T_2$ τέτοια ώστε
         το $G = T_1 \cup T_2$ να περιέχει υπογράφημα $H$ με
         $\delta(H) \geq 4$.

         Τότε $m(H) \geq \frac{\delta(H) \cdot n(H)}{2} = 2n(H)$.

         Για το $H$ έχουμε $H = F_1 \cup F_2$ όπου $F_1 \subseteq_{\text{υπ}}
         T_1, F_2 \subseteq_{\text{υπ}} T2$ δηλαδή τα $F_1, F_2$ είναι δάση
         και έτσι $m(F_1) \leq n(F_1) -1, m(F_2) \leq n(F_2) - 1$.
         Έτσι $m(H) \leq m(F_1) + m(F_2) \leq n(F_1) + n(F_2) - 2$.

         Όμως $n(F_1) + n(F_2) = |V(F_1) \cup V(F_2)| + |V(F_1) \cap V(F_2)|
         \leq 2|V(F_1) \cup V(F_2)| = 2n(H)$.

         Συνεπώς $m(H) \leq 2n(H) -2$. Άτοπο γιατί πριν δείξαμε ότι
         $m(H) \geq 2n(H)$.

         Για να δείξουμε τώρα ότι $c = 4$ μπορούμε να δούμε το παράδειγμα
         στο Σχήμα \ref{fig2.10.1} όπου έχουμε δύο δέντρα $T_1, T_2$ με 4 κορυφές
         το καθένα για τα οποία ισχύει $\delta^*(T_1 \cup T_2) = 3$.

         \begin{figure}
         \begin{center}
            \begin{tikzpicture}
               [nsblack/.style={scale=0.8,circle,draw=black!50,fill=black!10,thick}];

               \node[nsblack] (4) at (0, 0) {4};
               \node[nsblack] (3) at (3, 0) {3};
               \node[nsblack] (2) at (3, 3) {2};
               \node[nsblack] (1) at (0, 3) {1};

               \draw[black] [-] (1) -- (2);
               \draw[black] [-] (2) -- (3);
               \draw[black] [-] (3) -- (4);
               \draw[red] [-] (1) -- (3);
               \draw[red] [-] (2) -- (4);
               \draw[red] [-] (1) -- (4);
            \end{tikzpicture}
         \caption{Οι μαύρες ακμές ανήκουν στο $T_1$ και οι κόκκινες στο
                  $T_2$.}
         \label{fig2.10.1}
         \end{center}
         \end{figure}
      \end{proof}

   \item[2.10 $(\star)$]
      Σε κάθε δέντρο με $n$ κορυφές και διάμετρο τουλάχιστον
      $2k - 3$ υπάρχουν τουλάχιστον $n-k$ διαφορετικά μονοπάτια
      μήκους $k$.

      \begin{proof}

      Έστω $u, v$ δύο αντιδιαμετρικοί κόμβοι με $d(u, v) = \text{diam}(u, v)$
      και έστω $P$ το μονοπάτι που τους ενώνει.
      Ονομάζουμε $w$ την κορυφή πάνω στο $P$ που απέχει $d(u, w) = k-1$
      από την $u$. Τέτοια κορυφή υπάρχει αφού $|P| = d(u, v) \geq 2k-3 > k-1$.

      %Υπάρχει κόμβος $w$ πάνω στο
      %$P$ τέτοιος ώστε είτε $d(u, w) \geq k-1$ είτε $d(w, v) \geq k-1$
      %(διαφορετικά θα είχαμε $\text{diam}(G) = d(u, v) = d(u, w) + d(w, v) \leq
      %2(k-2) = 2k - 4$).

      %Υποθέτουμε λοιπόν χωρίς βλάβη της γενικότητας ότι $d(u, w) \geq k-1$
      %και μάλιστα επιλέγουμε το κοντινότερο τέτοιο $w$ στο $u$, δηλαδή
      %$d(u, w) = k-1$.
      
      Στο μονοπάτι $P'$ από $u$ στον $w$ υπάρχουν ακριβώς $k$ κορυφές.
      Θα δείξουμε ότι με αφετηρία κάθε μία από τις υπόλοιπες $n-k$ κορυφές
      μπορούμε να δημιουργήσουμε διαφορετικά μονοπάτια μήκους $k$.

      Έστω μια κορυφή $x$ που δεν ανήκει στο $P'$. Θα δημιουργήσουμε ένα
      μονοπάτι $T_x$ μήκους $k$ διακρίνοντας δύο περιπτώσεις:

      \begin{enumerate}
         \item Αν $w \in P(x, u)$ όπου $P(x, u)$ το μονοπάτι από $x$ σε
               $w$ στο δέντρο τότε θέτουμε $T_x$ το πρόθεμα μήκους $k$ του
               μονπατιού (δηλαδή το $T_x$ περιλαμβάνει την αφετηρία $x$ και
               τους επόμενους $k-1$ κόμβους).

               Τέτοιο πρόθεμα υπάρχει πάντα γιατί το $P(x, u) \geq
               P(x, w) + 1 = k$.

         \item Αν $w \notin P(x, u)$ τότε θεωρούμε το μονοπάτι $P(x, v)$
               για το οποίο ισχύει $w \in P(x, v)$ και θέτουμε $T_x$ το
               πρόθεμα μήκους $k$ αυτού του μονοπατιού.

               Όπως πριν, εξασφαλίζουμε την ύπαρξη τέτοιου προθέματος
               από το γεγονός ότι $P(x, v) \geq P(w, v) + 2 \geq
               (2k-3) - (k-1) + 2 = k$
      \end{enumerate}

      Τα παρακάτω λήμματα μας εξασφαλίζουν ότι τα μονοπάτια που προκύπτουν
      από την παραπάνω διαδικασία είναι όλα διαφορετικά μεταξύ τους.

      \begin{lm}
         Έστω δύο μονοπάτια $P_1, P_2$ σε ένα δέντρο που έχουν προκύψει ώς
         πρόθεμα (προσανατολισμένων) μονοπατιών από την κορυφή $x_1$ στην $u$ και
         από την $x_2$ στην $u$ αντίστοιχα όπου οι $x_1, x_2, u$ διαφορετικές
         μεταξύ τους κορυφές. Τότε $P_1 \neq P_2$\footnote{Ορίσαμε τα $P_1, P_2$
         ώς πρόθεμα προσανατολισμένων μονοπατιών, δηλαδή μονοπατιών
         με συγκεκριμένη αφετηρία και πέρας όμως από τη στιγμή που τα ορίζουμε
         τα θεωρούμε πλέον μη-προσανατολισμένα και έτσι έχει νόημα η σύγκριση
         $P_1 \neq P_2$.}
      \end{lm}
      \begin{proof}
         Από τον ορισμό των $P_1, P_2$ βλέπουμε ότι ο μόνος τρόπος να
         είναι το ίδιο μονοπάτι είναι αν έχουν ώς άκρα τις κορυφές
         $x_1, x_2$.

         Αυτό σημαίνει ότι $x_2 \in P(x_1, u), x_1 \in P(x_2, u)$ το οποίο
         είναι άτοπο άρα $P_1 \neq P_2$.
      \end{proof}

      \begin{lm}
      Έστω δύο μονοπάτια $T_{x_1}, T_{x_2}$ για $x_1 \neq x_2$ που έχουν προκύψει
      από τις περιπτώσεις (α'), (β') αντίστοιχα. Τότε $T_{x_1} \neq T_{x_2}$.
      \end{lm}

      \begin{proof}
      Έχουμε ότι
      $x_2 \notin P(x_1, u)$ γιατί διαφορετικά είτε θα είχαμε
      $x_2 \in P(x_1, w)$ και τότε η $x_2$ θα ήταν στην περίπτωση (α') είτε
      $x_2 \in P(w, u) = P'$ το οποίο
      δεν μπορεί να συμβαίνει αφού οι κορυφές του $P'$ δεν είναι αφετηρίες
      μονοπατιών.

      Συνεπώς, το $T_{x_1}$ που είναι υποσύνολο του $P(x_1, u)$ δεν μπορεί
      να περιέχει την $x_2$, άρα τα $T_{x_1}, T_{x_2}$ έχουν τουλάχιστον
      μία κορυφή διαφορετική και έτσι είναι διαφορετικά.
      \end{proof}

      Σε κάθε περίπτωση λοιπόν τα $n-k$ μονοπάτια που δημιουργήσαμε
      είναι όλα διαφορετικά μεταξύ τους.

      \begin{figure}
      \begin{center}
      \begin{tikzpicture}
         [black/.style={scale=0.7,circle,draw=black!50,fill=black!20,thick}];
         \node[black] (U) at (0, 0) {u};
         \node[black] (W) at (2, 0) {w};
         \node[black] (V) at (5, 0) {v};
         \node[black] (X1) at (2.5, 1) {$x_1$};
         \node[black] (X2) at (1.5, -1) {$x_2$};

         \draw [-] (U) .. controls (0.66, 0.5) and (1.2, -0.5) .. (W);
         \draw [-] (W) .. controls (3, 0.5) and (4, -0.5) .. (V);
         \draw [-] (X1) .. controls (3.2, 0.7) ..  (3.5, 0);
         \draw [-] (X2) .. controls (1.05, -0.3) .. (1, 0);

         \draw (1, 0) ellipse [x radius=1.8cm, y radius=0.5cm];
      \end{tikzpicture}
      \end{center}
      \caption{Οι καμπύλες γραμμές αναπαριστούν μονοπάτια. Η κορυφή $x_1$
      ανήκει στην περίπτωση (α') και η $x_2$ στην περίπτωση (β'). Σε κύκλο
      βρίσκονται οι κορυφές του μονοπατιού $P(u, w)$ από τις οποίες \emph{δεν}
      δημιουργούμε μονοπάτια.}
      \end{figure}

      \end{proof}
\end{enumerate}

\section{Συνεκτικότητα}
\begin{enumerate}
   \item[3.9 $(\star)$]
      Ένα γράφημα είναι δισυνεκτικό αν και μόνο αν μπορεί να κατασκευαστεί
      αρχίζοντας από το $K_3$ και εφαρμόζοντας μία ακολουθία μετασχηματισμών
      που μπορεί να είναι,

      \begin{itemize}
         \item Υποδιαίρεση ακμής.
         \item Πρόσθεση ακμής.
      \end{itemize}

      \begin{proof}
         Θα δούμε τις δύο κατευθύσεις του θεωρήματος ξεχωριστά.
         \begin{itemize}
            \item $\Leftarrow$
               Το $K_3$ είναι δισυνεκτικό άρα θα πρέπει να δείξουμε ότι
               οι παραπάνω μετασχηματισμοί διατηρούν αναλλοίωτη τη
               συνεκτικότητα.

               Πράγματι:

               \begin{itemize}
               \item
               Με την προσθήκη ακμής όλα τα μονοπάτια που υπάρχαν στο
               αρχικό γράφημα διατηρούνται. Έτσι, όσα εσωτερικώς διακεκριμένα
               μονοπάτια υπήρχαν μεταξύ ζευγών κορυφών συνεχίζουν να
               υπάρχουν και έτσι από το Θεώρημα Menger έχουμε ότι το
               γράφημα θα συνεχίσει να είναι δισυνεκτικό.

               \item
               Για την υποδιαίρεση ακμής, θα πρέπει να βεβαιωθούμε ότι
               δεν μπορούμε να αποσυνδέσουμε το γράφημα με την αφαίρεση
               μία κορυφής και συγκεκριμένα της κορυφής που βάλαμε με
               την υποδιαίρεση.

               Η αφαίρεση της νέας κορυφής ισοδυναμεί με αφαίρεση της
               υποδιαιρούμενης ακμής στο αρχικό γράφημα. Έστω $\{u, v\}$
               αυτή η ακμή. Επειδή το γράφημα
               αρχικά ήταν δισυνεκτικό, θα υπάρχει τουλάχιστον άλλο ένα
               μονοπάτι από την $u$ προς την $v$ άρα το γράφημα παραμένει
               συνεκτικό και μετά την αφαίρεση της $\{u, v\}$.
               \end{itemize}

            \item $\Rightarrow$
               Θα δείξουμε ότι αν ένα γράφημα $G$ είναι δισυνεκτικό τότε
               είτε:
               \begin{enumerate}
               \item[(1)] θα είναι το $K_3$, είτε
               \item[(2)] θα περιέχει μια κορυφή
               βαθμού 2 της οποίας οι γείτονες να μην είναι συνδεδεμένοι
               απευθείας
               (θα καλούμε τέτοιες κορυφές μή-απλοϊδείς) και η διάλυσή
               της δημιουργεί δισυνεκτικό γράφημα, είτε
               \item[(3)] θα περιέχει
               μία ακμή της οποίας η αφαίρεση οδηγεί σε δισυνεκτικό
               γράφημα.
               \end{enumerate}

               Έτσι, για κάθε δισυνεκτικό γράφημα μπορούμε να εφαρμόσουμε
               μία ακολουθία από διαλύσεις κορυφών και αφαιρέσεις ακμών
               μέχρι να καταλήξουμε στο $K_3$ και η αντίστροφη διαδικασία
               είναι που μας παράγει το $G$ από το $K_3$ όπως ζητάει η
               εκφώνηση.

               Αν το γράφημα $G$ περιέχει μια ακμή της οποίας η αφαίρεση
               διατηρεί το γράφημα δισυνεκτικό τότε έχουμε τελειώσει γιατί
               ισχύει το (3). Επομένως αρκεί να εξετάσουμε την περίπτωση
               όπου για όλες τις ακμές $e \in E(G)$ ισχύει
               $\kappa(G \backslash e) < 2$.

               Παρατηρούμε ότι $\kappa(G) = 2$ γιατί διαφορετικά, έστω
               ότι $\kappa(G) \geq 3$ τότε με την αφαίρεση μιας ακμής
               η συνεκτικότητα δεν θα έπρεπε να πέφτει πάνω από μία
               μονάδα (Παρατήρηση 5.7 των σημειώσεων του μαθήματος),
               όμως υποθέσαμε ότι η αφαίρεση οποιαδήποτε ακμής
               οδηγεί σε συνεκτικότητα μικρότερη του 2, δηλαδή έχουμε
               μείωση της συνεκτικότητας κατά 2 που είναι άτοπο.

               Σύμφωνα με το Θεώρημα Halin (συγκεκριμένα με το
               αντιθετο-αντίστροφό του) έχουμε ότι $\delta(G) \leq \kappa(G)
               = 2$. Ο ελάχιστος βαθμός ενός δισυνεκτικού γραφήματος
               δεν μπορεί να είναι μικρότερος του 2, άρα $\delta(G) = 2$.
               Έστω λοιπόν $u$ μια κορυφή βαθμού 2 και έστω $x, y$ οι
               γείτονές τις.

               Έστω τώρα ότι $\{x, y\} \in E(G)$. Αν το γράφημα έχει μόνο
               3 κορυφές τότε είναι το $K_3$ και έχουμε τελειώσει.
               Διαφορετικά έστω ότι έχει τουλάχιστον άλλη μία κορυφή $w$
               η οποία συνδέεται στην $x$. Επειδή το γράφημα είναι δισυνεκτικό
               θα πρέπει η αφαίρεση της $x$ να μην το αποσυνδέει, συνεπώς
               θα πρέπει να υπάρχει μονοπάτι $P$ από την $w$ στην
               $y$ που να μην χρησιμοποιεί την κορυφή $x$. Tότε όμως
               μεταξύ της $x$ και της $y$ θα υπήραν 3 εσωτερικά διακεκριμένα
               μονοπάτια: 1) $(x, y)$, 2) $(x, u, y)$, 3) $P$. Άτοπο
               γιατί τώρα η αφαίρεση της $\{x, y\}$ διατηρεί το γράφημα
               δισυνεκτικό.\footnote{Υπάρχει περίπτωση το $P$ να χρησιμοποιεί
               την $u$ ώς ενδιάμεσο κόμβο. Σε αυτή την περίπτωση θεωρούμε
               το $P'$ από το $w$ στο $u$ και δείχνουμε ότι υπάρχουν
               3 εσωτερικά διακεκριμένα μονοπάτια από το $x$ στο $u$.}

               Άρα η $u$ είναι μη απλοϊδής κορυφή βαθμού 2 και μένει να
               δείξουμε ότι η διάλυσή της διατηρεί τη συνεκτικότητα. Αυτό
               προκύπτει από το θεώρημα Menger αφού ό,τι μονοπάτια υπήρχαν
               πριν μεταξύ κορυφών συνεχίζουν να υπάρχουν.
         \end{itemize}
      \end{proof}

   \item[3.10 $(\star\star)$]
Για κάθε $k$ κορυφές ενός $k$-συνεκτικού γραφήματος, υπάρχει κύκλος που να τις περιέχει όλες.
\begin{proof}
Έστω $k$ κορυφές του γραφήματος $G$ και $C$ κύκλος που περιέχει όσο το δυνατόν περισσότερες από τις $k$ κορυφές. Έστω $S$ το σύνολο των $k$ κορυφών. Αν ο $|C|$ περιέχει και τις $k$, τελειώσαμε. 
Διαφορετικά, περιέχει μόνο $l$ από αυτές και έστω $u$ μία από τις 
$k$ κορυφές, η οποία δεν 
βρίσκεται στον κύκλο. Από το Λήμμα 1, υπάρχουν $min(|C|,k)$ εσωτερικώς διακεκριμένα μονοπάτια από το $u$ προς τις κορυφές του κύκλου, και κανένα δεν τελειώνει στην ίδια κορυφή του κύκλου. Έστω $v_i$ μία
απαρίθμηση των κορυφών του κύκλου (με τη σειρά που εμφανίζονται πάνω στον κύκλο) οι οποίες αποτελούν άκρο κάποιου μονοπατιού από τα παραπάνω 
και $P_i$ τα αντίστοιχα μονοπάτια. Επίσης έστω $F_i$ το μονοπάτι από 
την $v_i$ στην $v_{i+1}$ το οποίο δεν περιέχει καμία άλλη από τις $v_j$. (Έχουμε θεωρήσει ότι $v_{min(|C|,k)+1}\equiv v_1$). Αν ο κύκλος έχει μήκος $l$, τότε περιέχει μόνο κορυφές από το $S$. Ο κύκλος
$v_1, P_1, u, P_2, v_2, v_3, ..., v_l, v_1$ περιέχει $l+1$ στοιχεία του $S$, άτοπο. Αν έχει μήκος $>l$, τότε οι κορυφές $v_i$ είναι τουλάχιστον $l+1$. Αυτό σημαίνει ότι υπάρχουν τουλάχιστον $l+1$
διαφορετικά μονοπάτια $F_i$. Άρα θα υπάρχει ένα $F_i$ το οποίο δεν περιέχει στο εσωτερικό του καμία κορυφή του $S$. Τότε, ο κύκλος $v_1, F_1, v_2, ..., v_i, P_i, u, P_{i+1}, v_{i+1}, F_{i+1}, ...,v_1$ 
έχει $l+1$ στοιχεία του $S$, άτοπο.
Άρα για κάθε σύνολο $k$ κορυφών, υπάρχει κύκλος που τις περιέχέι όλες.
\end{proof}

Λήμμα 1: Έστω $k$-συνεκτικό γράφημα, κύκλος του με τουλάχιστον $l$ κορυφές με $l<k$ και τυχαία κορυφή $u$ εκτός του κύκλου. Τότε υπάρχουν $l$ κορυφές του κύκλου $v_1, v_2, ..., v_l$ και 
εσωτερικώς διακεκριμένα μονοπάτια $P_i = u...v_i$ για κάθε $1\leq i\leq l$.
\begin{proof}
Έστω μία νέα κορυφή $v$ που συνδέεται με ακμή με όλες τις κορυφές του κύκλου. Δηλαδή θεωρούμε γράφημα $G$ με $V(G')=V(G)\cup \{v\}$ και $E(G')=E(G)\cup \{(v,x)|x\in C\}$. Το $G$ είναι $l$-συνεκτικό:
Αν σβήσουμε $l-1$ κορυφές και σε αυτές περιέχεται η $v$, τότε οι κορυφές που απομένουν συνδέονται λόγω της $k$-συνεκτικότητας του αρχικού γραφήματος. Σε διαφορετική περίπτωση, θα σβηστούν το πολύ $l-1$
κορυφές του κύκλου και συνεπώς θα μείνει τουλάχιστον μία άκμή από την $v$ προς μια κορυφή του κύκλου, άρα το γράφημα θα παράμείνει συνεκτικό. Αφού το γράφημα είναι $l$-συνεκτικό, θα υπάρχουν $l$ εσωτερικώς
διακεκριμένα μονοπάτια από την κορυφή $u$ στην κορυφή $v$. Κάθε ένα από αυτά τα μονοπάτια περνάει από τουλάχιστον μία κορυφή του κύκλου. Για κάθε μονοπάτι $P = u...v$, θεωρούμε την πρώτη φορά που περνάει
από μία κορυφή του κύκλου. Έστω ότι αυτή είναι η $x_i$. Το σύνολο των μονοπατιών $\{P_i = u...x_i\}$ είναι το ζητούμενο, αφού τα μονοπάτια είναι εσωτερικώς διακεκριμένα και καταλήγουν σε $l$ διαφορετικές
κορυφές του κύκλου.

\end{proof}

\end{enumerate}
\section{Εμβαπτίσεις}
\begin{enumerate}
\item[4.6 $(\star)$]
   Έστω ενεπίπεδο γράφημα $\Gamma$ και έστω $\Gamma^*$ το δυικό του. Δείξτε
   ότι τα $\Gamma$ και $\Gamma^*$ έχουν το ίδιο πλήθος δεντροπαραγόντων.
   \begin{proof}

      Θα δείξουμε ότι υπάρχει συνάρτηση $f$ 1-1 και επί από το σύνολο των
      δεντροπαραγόντων του $\Gamma$ στο σύνολο των δεντροπαραγόντων
      του $\Gamma^*$ και συνεπώς τα δύο σύνολα θα έχουν το ίδιο πλήθος
      στοιχείων.

      Ως γνωστόν, το δυικό ενός γραφήματος έχει το ίδιο πλήθος ακμών
      με το αρχικό και μάλιστα κάθε ακμή $e$ του αρχικού αντιστοιχεί σε
      εκείνη την ακμή $e^*$ του δυικού η οποία συνδέει τις δύο όψεις τις
      οποίες ``βλέπει'' η $e$.

      Έστω ένας δεντροπαράγοντας $T$ του $\Gamma$. Δημιουργούμε ένα
      υπογράφημα $T^*$ του $\Gamma^*$ κρατώντας όλες τις ακμές $e^*$
      των οποίων οι αντίστοιχες $e$ στο $\Gamma$ δεν ανήκουν στο $T$,
      δηλαδή $E(T^*) = \{ e^*\ |\ e \notin T \}$.

      Θα δείξουμε ότι το $T^*$ είναι δεντροπαράγοντας και η αντιστοιχία
      είναι όντως 1-1 και επί. Το δεύτερο φαίνεται εύκολα αφού ένας
      δεντροπαράγοντας χαρακτηρίζεται από το σύνολο των ακμών που περιέχει
      και έχουμε ήδη δείξει ότι υπάρχει 1-1 και επί αντιστοιχία των ακμών
      του $\Gamma$ με τις ακμές του $\Gamma^*$.

      Για το πρώτο, θα χρησιμοποιήσουμε ένα λήμμα που συνδέει τους κύκλους
      ενός επίπεδου γραφήματος με τις τομές (cuts) του δυϊκού και αντιστρόφως.
      
      \begin{defn}
         Με τον όρο \emph{τομή} (cut) μιας επίπεδης απεικόνισης ενός γραφήματος
         $G$ εννούμε μια κλειστή καμπύλη γραμμή που δεν τέμνει τις κορυφές
         του $G$ και περιέχει τουλάχιστον μία κορυφή στο εσωτερικό της
         και τουλάχιστον μία στο εξωτερικό της.
      \end{defn}

      \begin{lm}
         Έστω επίπεδο γράφημα $G$, και έστω $G^*$ το δυϊκό του για μία επίπεδη
         απεικόνιση του $G$. Κάθε \emph{κύκλος} $C^*$ (όχι απαραίτητα απλός)
         του δυϊκού γραφήματος αντιστοιχεί σε μια \emph{τομή}
         $C$ στο αρχικό γράφημα $G$ και αντιστρόφως.
         Επιπλέον το πλήθος των ακμών του $G$ που διαπερνούν την τομή $C$,
         είναι ίσο με το μήκος του κύκλου $C^*$.
      \end{lm}
      \begin{proof}
         Με βάση μία επίπεδη απεικόνιση του $G$ σχεδιάζουμε το δυϊκό γράφημα $G^*$ ως εξής:

         \begin{itemize}
            \item Για κάθε όψη $f_i$ του $G$ επιλέγουμε ένα εσωτερικό της σημείο $v_i^*$ το οποίο αναπαριστά
                  την κορυφή του δυϊκού που αντιστοιχεί στην όψη αυτή.
         
            \item
                  Για κάθε ακμή $e_i$
                  του αρχικού γραφήματος, η οποία βρίσκεται στο περιθώριο δύο όψεων $f_i, f_j$
                  (όχι απαραίτητα διαφορετικών μεταξύ τους) προσθέτουμε μια καμπύλη γραμμή
                  μεταξύ των κορυφών $v_i^*, v_j^*$ του δυϊκού που αναπαριστά την ακμή $e_i^*$ του
                  δυϊκου και η οποία τέμνει τη ακμή $e_i$.
         \end{itemize}

         Είναι τώρα φανερό ότι ένας κύκλος $C^*$ στο δυϊκό γράφημα
         αποτελεί μια κλειστή καμπύλη η οποία έχει εσωτερικό και εξωτερικό μέρος άρα
         θα είναι μια τομή για το αρχικό γράφημα. Επιπλέον κάθε ακμή του κύκλου
         $C^*$ τέμνει ακριβώς μία ακμή του αρχικού γραφήματος και έτσι υπάρχει
         1-1 αντιστοιχία των ακμών του κύκλου και αυτών που διαπερνούν
         την τομή.

         Αντίστοιχα, μια τομή του αρχικού γράφηματος θα είναι μια καμπύλη που
         θα διέρχεται από όψεις του γραφήματος διαπερνώντας ακμές, δηλαδή για
         το δυϊκό γράφημα θα είναι ένας κύκλος.
      \end{proof}

      Γυρνόντας τώρα πίσω στο υπογράφημα $T^*$, θα δείξουμε κατ' αρχάς ότι
      είναι δέντρο. Πράγματι, έστω ότι το $T^*$ περιείχε κύκλο. Τότε αυτό
      σημαίνει ότι στο $T$ θα υπήρχε μία τομή που διαχωρίζει τις κορυφές
      του και οι ακμές που διαπερνάνε την τομή δεν ανήκουν στο $T$. Αυτό
      όμως είναι άτοπο γιατί τότε το $T$ δεν θα ήταν συνδεδεμένο.

      Με εντελώς ανάλογο επιχείρημα μπορούμε να δείξουμε και ότι το $T^*$
      είναι συνδεδεμένο. Πράγματι, έστω ότι τουλάχιστον 2 συνεκτικές συνιστώσες
      στο $T^*$, τότε θα μπορούσαμε να τις διαχωρίσουμε με μία τομή την οποία
      θα διαπερνούσαν ακμές $e^* \notin T$, οι οποίες να δημιουργούσαν
      κύκλο στο $T$ με ακμές $e \in T$ το οποίο είναι άτοπο γιατί το 
      $T$ είναι δέντρο και δεν μπορεί να περιέχει κύκλους.
   \end{proof}

\item[4.9 $(\star\star)$]
	Ορίζουμε το τετράγωνο $G^2$ ενός γραφήματος ως εξής: $G^2 = (V(G), \{(x,y) | dist_G (x,y) \leq 2\})$. Περιγράψτε πλήρως όλα τα γραφήματα $G$ για τα οποία το $G^2$ είναι επίπεδο.
	\begin{proof}
Για να είναι το $G^2$ επίπεδο, θα πρέπει να μην περιέχει κανένα εκ των $K_5$ και $K_{3,3}$ ως ελάσσον. Αν υπάρχει στο $G$ κορυφή με βαθμό τουλάχιστον 4, όλοι οι γείτονές της έχουν απόσταση 2, άρα συνδέονται
με ακμή στο $G^2$, δηλαδή το $G^2$ περιέχει σαν ελάσσον το $K_5$, άτοπο. Άρα $\Delta (G)\leq 3$. Έστω μια κορυφή τομής του $G$. Όπως είπαμε ο βαθμός της θα είναι το πολύ 3, άρα δεν μπορεί να είναι κοινή
κορυφή δύο δισυνεκτικών συνιστωσών που δεν είναι το $K_2$.
 Συνεπώς αν μια κορυφή ανήκει σε μια δισυνεκτική συνιστώσα που δεν είναι το $K_2$, έχει το πολύ μια ακμή που δεν ανήκει στη συνιστώσα, που είναι και γέφυρα στο $G$. Από το Λήμμα 1,
μπορούμε να θεωρήσουμε ότι το γράφημά μας (έστω $H$) είναι ένα δισυνεκτικό γράφημα $W$ στο οποίο έχουμε προσθέσει επιπλέον ακμές, τέτοιες ώστε το άκρο τους που δεν ανήκει στο $W$ να έχει βαθμό 1.
Στη συνέχεια θα αποδείξουμε ότι το $W$ αποτελεί έναν κύκλο. Ας υποθέσουμε διαφορετικά: Έστω $C$ ένας μέγιστος κύκλος και $v$ μία κορυφή του $W$ που δεν ανήκει σε αυτόν. Όπως έχουμε αποδείξει στο Λήμμα 1
της 3.10, υπάρχουν δύο εσωτερικώς διακεκριμένα μονοπάτια από την $v$ προς δύο διαφορετικές κορυφές του $C$, έστω $x$ και $y$. Τότε ορίζονται 3 εσωτερικώς διακεκριμένα μονοπάτια από την $x$ στην $y$: Δύο
πάνω στον κύκλο και ένα που περνάει από την $v$. Σύμφωνα με το Λήμμα 3, όμως, αυτό σημαίνει ότι το $C\cup \{v\}$ είναι το $K_4$, το οποίο με τη σειρά του σημαίνει ότι ο $C$ είναι τρίγωνο. Αυτό είναι άτοπο,
διότι έτσι σχηματίζεται μεγαλύτερος κύκλος (μήκους τουλάχιστον 4) αν συμπεριλάβουμε την $v$ και τα μονοπάτια της προς τις $x$, $y$. Άρα το $W$ αποτελεί κύκλο. Από το Λήμμα 2, αυτός ο κύκλος θα πρέπει επίσης
να είναι άρτιος. Η τελευταία αναγκαία συνθήκη για να είναι το $G^2$ επίπεδο είναι να μην υπάρχει στον $G$ τρίγωνο, και οι τρεις κορυφές του οποίου να είναι κορυφές τομής. Παρατηρούμε, όντως, ότι αν έχουμε
ένα τρίγωνο, κάθε κορυφή του οποίου συνδέεται με μία ακμή με έναν κόμβο βαθμού 1, η απόσταση των κορυφών του τριγώνου από τις κορυφές βαθμού 1 είναι το πολύ 2, άρα το τετράγωνό του είναι το $K_{3,3}$.
Συνεπώς το $G^2$ δεν είναι επίπεδο, άτοπο.

\newline\newline
Συνοψίζουμε τις τρεις αναγκαίες συνθήκες που έχουμε για το $G$, έτσι ώστε το $G^2$ να είναι επίπεδο:
α) Ο βαθμός κάθε κορυφής είναι το πολύ 3.
β) Κάθε δισυνεκτική συνιστώσα με τουλάχιστον 5 κορυφές είναι κύκλος άρτιου μήκους.
γ) Δεν υπάρχει τρίγωνο, του οποίου όλες οι κορυφές είναι κορυφές τομής.

\newline\newline
Στη συνέχεια θα αποδείξουμε ότι αυτές οι συνθήκες είναι και ικανές: Έστω το γράφημα $H$, το οποίο είναι ένα δισυνεκτικό γράφημα $W$ στο οποίο έχουμε προσθέσει επιπλέον ακμές, τέτοιες ώστε το άκρο που
δεν ανήκει στο $W$ να έχει βαθμό 1. Αν το $H$ έχει το πολύ 4 κορυφές το τετράγωνό του 
είναι προφανώς επίπεδο. Αν το $W$ είναι τρίγωνο, τουλάχιστον μία από τις κορυφές έχει βαθμό 2. Άρα έχουμε 5 κορυφές, αλλά η απόσταση 
μεταξύ των δύο κορυφών που δεν ανήκουν στο τρίγωνο είναι >2, άρα το τετράγωνο αυτού του γραφήματος δεν είναι το $K_5$, οπότε είναι επίπεδο. Αν το $W$ έχει 4 κορυφές, 
τότε έχουμε δύο περιπτώσεις ακραίων γραφημάτων (ως προς τις ακμές) των οποίων εύκολα βλέπουμε ότι τα τετράγωνα είναι επίπεδα. Ομοίως αν το $W$ είναι το $K_2$. Τώρα, αν το $W$ είναι ένας άρτιος κύκλος
με τουλάχιστον 6 κορυφές, η ακραία περίπτωση είναι κάθε κορυφή του να έχει και μία ακμή προς κάποια κορυφή με βαθμό 1. Η εμβάπτιση σε αυτή την περίπτωση είναι πολύ παρόμοια με την περίπτωση που το $W$
είναι κύκλος μήκους 4. Σε κάθε περίπτωση το τετράγωνο του γραφήματος που πληροί τις παραπάνω προϋποθέσεις είναι επίπεδο, άρα οι συνθήκες είναι αναγκαίες και ικανές.

	\end{proof}

Λήμμα 1: Έστω δισυνεκτικά γραφήματα $H_1$, $H_2$ χωρίς κοινές κορυφές μεταξύ τους και $e$ μια γέφυρα που τα συνδέει. Τότε το τετράγωνο του γραφήματος που προκύπτει είναι επίπεδο αν και
μόνο αν τα τετάγωνα των γραφημάτων $H_1 \cup e$ και $H_2 \cup e$ είναι επίπεδα.
	\begin{proof}
Η μία κατεύθυνση είναι προφανής: αν το τετράγωνο του $H_1\cup H_2\cup e$ είναι επίπεδο, τότε και τα τετράγωνα των $H_1\cup e$ και $H_2\cup e$ είναι επίπεδα, αφού δεν μπορούν παρά να έχουν λιγότερες ακμές.
Ανίστροφα τώρα, έστω $x$ το άκρο της $e$ που ανήκει στο $H_1$ και $y$ το άκρο της που ανήκει στο $H_2$. Οι $x$, $y$ έχουν βαθμό το πολύ 2 στα $H_1$ και $H_2$. 
Αν υπάρχει επίπεδη εμβάπτιση του τετραγώνου του $H_1\cup e$, τότε υπάρχει επίπεδη εμβάπτισή του στην οποία η ακμή $e$ βρίσκεται στην εξωτερική όψη. 
Ομοίως και για το $H_2\cup e$, οπότε αν ενώσουμε τις δύο εμβαπτίσεις στην ακμή $e$, καταλήγουμε σε μία επίπεδη εμβάπτιση του τετραγώνου του $H_1\cup H_2\cup e$.
	\end{proof}
Λήμμα 2: Κάθε περιττός κύκλος μήκους τουλάχιστον 5 έχει μη επίπεδο τετράγωνο.
	\begin{proof}
Αρχικά, στον κύκλο μήκους 5 όλες οι ανά δύο αποστάσεις των κορυφών είναι το πολύ δύο, άρα το τετράγωνό του είναι το $K_5$, δηλαδή δεν είναι επίπεδο. Θεωρούμε τώρα κύκλο περιττού μήκους τουλάχιστον 7.
Έστω $v_1, v_2, ..., v_{2k}, v_{2k+1}, v_1$ μία απαρίθμηση των κορυφών του κύκλου με τη σειρά. Θα δείξουμε ότι το τετράγωνο αυτού του κύκλου περιέχει το $K_{3,3}$ ως ελάσσον,
άρα δεν είναι επίπεδο. Θεωρούμε τα δύο σύνολα $\{v_1,v_4,v_5\}$ και $\{v_2,v_3,v_6\}$. Στο τετράγωνο του γραφήματος, τα παρακάτω ζεύγη κορυφών συνδέονται με ακμή: $(v_1, v_2)$, $(v_1,v_3)$, $(v_4,v_2)$,
$(v_4,v_3)$, $(v_4,v_6)$, $(v_5,v_3)$, $(v_5,v_6)$. Θεωρούμε τα μονοπάτια $P_1=v_1 v_{2k} v_{2k-2} ... v_6$ και $P_2=v_2 v_{2k+1} v_{2k-1} ... v_7 v_5$. Αυτά τα μονοπάτια είναι εσωτερικώς διακεκριμένα και 
συνδέουν τα ζευγάρια $(v_1,v_6)$ και $(v_5,v_2)$. Η σύνθλιψη αυτών των μονοπατιών και η διαγραφή των περισσευούμενων ακμών έχει ως αποτέλεσμα το $K_{3,3}$, αφού υπάρχει ακμή ανάμεσα σε κάθε δύο κορυφές
στα δύο άκρα της διαμέρισης $\{v_1,v_4,v_5\}$ και $\{v_2,v_3,v_6\}$. Συνεπώς το γράφημα δεν είναι επίπεδο.
	\end{proof}
Λήμμα 3: Έστω γράφημα $G$, το οποίο αποτελείται από 3 εσωτερικώς διακεκριμένα μονοπάτια μεταξύ δύο κορυφών $u$ και $v$. Αν το $G^2$ είναι επίπεδο, τότε το $G$ είναι το $K_4$ χωρίς μία ακμή, δηλαδή τα 
δύο μονοπάτια έχουν μήκος 2 και το άλλο έχει μήκος 1.
	\begin{proof}
Έστω $P_1$, $P_2$, $P_3$ τα τρία μονοπάτια, και $k_1\leq k_2\leq k_3$ αντίστοιχα τα μήκη τους. (Σημείωση: Δεν έχουμε πολυγράφημα). Έστω $k_1=1$. Τότε $k_2,k_3\geq 2$ και, αν $k_2=k_3=2$, τότε 
έχουμε το $K_4$ χωρίς μία ακμή. Έστω τώρα $k_3>2$. Θεωρούμε $u$ και $v$ τα κοινά άκρα των μονοπατιών. Επίσης έστω $x$ η κοντινότερη στο $u$ κορυφή του $P_3$, $y$ η κοντινότερη στο $v$ 
κορυφή του $P_3$ ($x\neq y$), $z$
η κοντινότερη στο $u$ κορυφή του $P_2$ και $w$ η κοντινότερη στο $v$ κορυφή του $P_2$ (Μπορεί και $z\equiv w$. 
Θα δείξουμε ότι το τετράγωνο του εν λόγω γραφήματος έχει σαν ελάσσον το $K_5$. Αφού $k_1=1$, στο τετράγωνο του γραφήματος υπάρχει ακμή μεταξύ των εξής ζευγαριών:
$(x,u)$, $(x,z)$, $(x,v)$, $(u,z)$, $(u,v)$, $(u,y)$, $(z,v)$, $(v,y)$. Τώρα, τα $x$ και $y$ συνδέονται με το μονοπάτι $P_3$, ενώ η $y$ συνδέεται με την $w$ και αυτή με την $z$ μέσω του $P_2$. Τα
μονοπάτια που παραθέσαμε είναι ανά δύο εσωτερικώς διακεκριμένα και συνδέουν κάθε ζευγάρι από τις 5 κορυφές. Συνεπώς αν συνθλίψουμε αυτά τα μονοπάτια και διαγράψουμε τις περισσευούμενες ακμές,
καταλήγουμε στο $K_5$. Άρα αν $k_1=1$, θα πρέπει $k_2=k_3=2$. 

Έστω $k_1\geq 2$. Αν $k_2=2$, έστω $x$ η εσωτερική κορυφή του $P_1$, $y$ η εσωτερική κορυφή του $P_2$ και $z$ η εσωτερική κορυφή του $P_3$ που βρίσκεται πιο κοντά στο $u$.
Τα ζευγάρια $(z,u)$, $(z,x)$, $(z,y)$, $(u,x)$, $(u,y)$, $(u,v)$, $(x,v)$, $(y,v)$, $(x,y)$ έχουν ακμή στο τετράγωνο του γραφήματος. Συνθλίβουμε το μονοπάτι μεταξύ των $z$ και $v$ (μέρος του $P_3$) και
παίρνουμε το $K_5$. Άρα $k_2\geq 3$.
Τα $k_i$ πρέπει να έχουν και τα τρία το ίδιο υπόλοιπο mod 2, διότι σε διαφορετική περίπτωση
θα σχηματιζόταν περιττός κύκλος, το οποίο, όπως δείξαμε στο Λήμμα 2, σημαίνει ότι το τετράγωνο του γραφήματος δεν είναι επίπεδο. Αν είναι και τα τρία άρτια, έστω $x\equiv x_1$,...,$x_{k_1-1}\equiv x'$ 
οι εσωτερικές κορυφές του $P_1$, $y\equiv y_1$,...,$y_{k_2-1}\equiv y'$ οι εσωτερικές κορυφές του $P_2$ και $z\equiv z_1$,...,$z_{k_3-1}\equiv z'$ οι εσωτερικές κορυφές του $P_3$. 
(Όλα ξεκινώντας από την $u$ προς την $v$). Θα δείξουμε ότι στο τετράγωνο αυτού του γραφήματος περιέχεται το $K_{3,3}$ ως ελάσσον. Θεωρούμε τις κορυφές ${u,v,x,z,x',z'}$ και τη 2-διαμέρισή τους
$(u,x,z')$, $(v,x',z)$. Παρατηρούμε ότι υπάρχουν ακμές στο τετράγωνο του γραφήματος ανάμεσα στις εξής κορυφές: $(u,z)$, $(x,z)$, $(v,z')$, $(x',z')$. Τώρα θεωρούμε τα μονοπάτια: 
$x x_2 x_4 ... x_{k_1 -2} v$, $x x_1 x_3 ... x_{k_1 -1}\equiv x'$, $u y_2 y_4 ... y_{k_2 -2} v$, $u y_1 y_3 ... y_{k_2-1} x_{k_1-1}\equiv x'$, $z\equiv z_1 z_2 ... z_{k_3-1}\equiv z'$. Αυτά τα μονοπάτια
είναι ανά δύο εσωτερικώς διακεκριμένα και η σύνθλιψή τους μας οδηγεί στο $K_{3,3}$. Η περίπτωση που τα $k_i$ είναι και τα 3 περιττά είναι εντελώς παρόμοια.

Συμπεραίνουμε ότι, αφού κάθε άλλη περίπτωση κατέληξε στη μη επιπεδότητα του $G^2$, ότι το $G$ είναι το $K_4$ χωρίς μία ακμή.
	\end{proof}

\item[4.10 $(\star \star)$]
Καλούμε $(x,y)$-τοροειδές πλέγμα το γράφημα $H_{x,y}$, όπου $V(H_{x,y})=\{0,...,x-1\}\times \{0,...,y-1\}$ και $E(H_{x,y})=\{((a,b),(c,d)) | |a-c\;mod\;x| + |b-d\;mod\;y|=1 \}$. Δείξτε ότι δεν υπάρχει $x$
τέτοιο ώστε το $2\cdot K_5$ να είναι τοπολογικό ελάσσον του $(x,y)$-τοροειδούς πλέγματος.

\begin{proof}
Σύμφωνα με το Λήμμα 2, κάθε $(x,y)$-τοροειδές πλέγμα είναι εμβαπτίσιμο στον τόρο. Έστω ένα τέτοιο τοροειδές πλέγμα.
Αν περιέχει το $2\cdot K_5$ ως ελάσσον, αυτό θα σημαίνει ότι και το $2\cdot K_5$ είναι εμβαπτίσιμο στον τόρο. Αρκεί λοιπόν
να δείξουμε ότι το $2\cdot K_5$ δεν είναι εμβαπτίσιμο στον τόρο. 
Αυτό το γράφημα έχει δύο δισυνεκτικές συνιστώσες και είναι και
οι δύο ισόμορφες με το $K_5$. Γνωρίζουμε όμως ότι το γένος του $K_5$ είναι τουλάχιστον 1, αφού δεν είναι επίπεδο. Συνεπώς, από το Λήμμα 1, $\gamma (G)=2\cdot \gamma(K_5)\geq 2$. Συνεπώς το $2\cdot K_5$
δεν είναι εμβαπτίσιμο στον τόρο, ο οποίος είναι μια επιφάνεια με γένος 1. Συμπεραίνουμε, λοιπόν, ότι δεν υπάρχουν $x$,$y$, έτσι ώστε το $2\cdot K_5$ να είναι ελάσσον (άρα και τοπολογικό ελάσσον) του 
$(x,y)$-τοροειδούς πλέγματος.
\end{proof}

Λήμμα 1: Έστω η αποσύνθεση ενός (συνεκτικού ή όχι) γράφήματος σε δισυνεκτικές συνιστώσες. Αν $\gamma(G)$ είναι το γένος ενός γραφήματος και $G_i$ οι δισυνεκτικές συνιστώσες του, τότε ισχύει ότι
$\gamma(G) = \sum {\gamma(G_i)}$.

\begin{proof}
Έχει αποδειχθεί από τους Battle, Harary, Kodama, Youngs στην εργασία Additivity of the genus of a graph.
\newline (https://projecteuclid.org/download/pdf\_1/euclid.bams/1183524922).
\end{proof}

Λήμμα 2: Το $(x,y)$-τοροειδές πλέγμα είναι εμβαπτίσιμο στον τόρο.
\begin{proof}
Το $(x,y)$-τοροειδές πλέγμα μπορεί εύκολα να εμβαπτιστεί
στον τόρο, ο οποίος είναι μια επιφάνεια με γένος 1: Σχεδιάζουμε το $(x,y)$-πλέγμα, το οποίο είναι επίπεδο, σε κάποια περιοχή του τόρου ισόμορφη με τον ανοιχτό δίσκο. 
Στη συνέχεια, από τις υπόλοιπες ακμές σχεδιάζουμε αυτές που είναι 
στην $y$-διάσταση στην περιφέρεια της διατομής του τόρου, και αυτές που είναι στην $x$ διάσταση κατά μήκος της περιμέτρου ολόκληρου του τόρου.
Το ζητούμενο αποδείχθηκε.
\end{proof}

\end{enumerate}

\section{Δομές σε γραφήματα}

\begin{enumerate}
   \item[5.9 $(\star)$]
      Κάθε γράφημα περιέχει τουλάχιστον $\frac{m(G)(4m(G) - n^2(G))}{3n(G)}$
      τρίγωνα.

      \begin{proof}

      Έστω μια ακμή $\{u, v\}$. Η ιδέα είναι να βρούμε το ελάχιστο πλήθος
      τριγώνων στα οποία μπορεί να ανήκει αυτή η ακμή και έτσι μετά αθροίζοντας
      κατάλληλα να μπορέσουμε να φράξουμε από κάτω το συνολικό πλήθος των
      τριγώνων του γραφήματος.

      Ορίζουμε $U = N_G(u) \backslash v,
      V = N_G(v) \backslash u$. Ισχύει ότι $|U| + |V| = d(u) + d(v) - 2$.
      Επίσης, $|U \cup V| \leq n(G) - 2$ αφού δεν υπάρχουν πάνω από
      $n(G)-2$ κορυφές που να απομένουν στο γράφημα.

      Άρα, έχουμε:

      \[ |U \cap V| = |U| + |V| - |U \cup V| \geq d(u) + d(v) - n(G) \]

      Κάθε κορυφή που ανήκει στο $U \cap V$ δημιουργεί τρίγωνο με τις
      κορυφές $u, v$. Άρα το πλήθος των τριγώνων $|T_{\{u, v\}}|$ που μπορεί
      να ανήκει
      η ακμή $\{u, v\}$ είναι τουλάχιστον $d(u) + d(v) - n(G)$.

      Αν συμβολίσουμε με $T$ το σύνολο των τριγώνων του $G$ έχουμε:

      \[ 3|T| = \sum_{\{u, v\} \in E(G)} T_{\{u, v\}} \]

      επειδή κάθε τρίγωνο περιέχει 3 ακμές.

      Συνεπώς:

      \begin{align*}
         |T| &\geq \frac13 \sum_{\{u, v\} \in E(G)} ( d(u) + d(v) - n(G) )\\
             &= \frac13 \sum_{\{u, v\} \in E(G)} ( d(u) + d(v) ) -
                \frac{n(G)m(G)}3 \\
             &= \frac13 \sum_{u \in V(G)} d^2(u) - \frac{n(G)m(G)}3\\
             &\geq \frac1{3n(G)} \left( \sum_{u \in V(G)} d(u) \right)^2
                 - \frac{n(G)m(G)}3\\
             &= \frac{4m^2(G)}{3n(G)} - \frac{n(G)m(G)}3\\
             &= \frac{m(G)(4m(G) - n^2(G))}{3n(G)}\\
      \end{align*}

      Όπου το 4ο βήμα προκύπτει από την ανισότητα Cauchy-Schwarz:

      \begin{align*}
         d(u_1)\cdot1 + d(u_2)\cdot2 + \ldots + d(u_n)\cdot1
         &\leq (d^2(u_1) + \ldots + d^2(u_n)) \cdot (1 + \ldots + 1)\\
         &= (d^2(u_1) + \ldots + d^2(u_n)) \cdot n
      \end{align*}

      \end{proof}

   \item[5.10 $(\star \star)$]
Δείξτε ότι ένα πολυγράφημα είναι σειριακό-παράλληλο αν είναι 2-συνεκτικό και δεν περιέχει καμία υποδιαίρεση του $K_4$ ως ελάσσον. Ένα γράφημα καλείται σειριακό-παράλληλο αν μπορεί
να προκύψει από το $K_2$ μετά από σειρά υποδιαιρέσεων ακμών ή διπλασιασμών ακμών (δηλαδή αντικατάσταση μιας ακμής από μια διπλής πολλαπλότητας με τα ίδια άκρα).
	\begin{proof}
Αρχικά, αν ένα γράφημα δεν περιέχει καμία υποδιαίρεση του $K_4$ ως ελάσσον, δεν περιέχει ούτε το $K_4$ ως ελάσσον.
Θα δείξουμε ότι αν ένα πολυγράφημα είναι 2-συνεκτικό και δεν περιέχει το $K_4$ ως ελάσσον, τότε είναι σειριακό-παράλληλο.
Για 2 κορυφές ισχύει, αφού έχουμε το $K_2$ που είναι σειριακό-παράλληλο.
Για 3 κορυφές επίσης ισχύει, αφού έχουμε το $K_3$, το οποίο μπορεί να προκύψει από την εξής ακολουθία κινήσεων: 
$K_2$->διπλασιασμός ακμής, υποδιαίρεση της μίας ακμής.

Θεωρούμε το γράφημα $G$ με τον ελάχιστο αριθμό κορυφών, το οποίο είναι 2-συνεκτικό, δεν περιέχει το $K_4$ ως ελάσσον και δεν είναι σειριακό-παράλληλο. Από το Λήμμα 1, το γράφημα $G$ δεν μπορεί
να είναι 3-συνεκτικό. 

Έστω ένας 2-διαχωριστής $u$, $v$ και $G'$ μία συνεκτική συνιστώσα που προκύπτει μετά τη διαγραφή των κορυφών $u$, $v$. Έστω γράφημα $H$ με 
$V(H)=V(G')\cup \{u,v\}$ και $E(H) = \{(x,y) | x\in V(H), y\in V(H), (x,y)\in G, (x,y)\neq (u,v)\}$. Το γράφημα $H$ είναι συνεκτικό, διότι σε διαφορετική περίπτωση κάποια από τις κορυφές 
$u$, $v$ θα αποτελούσε κορυφή τομής. 

Αρχικά θα αποδείξουμε ότι το $H$ δεν μπορεί να είναι 2-συνεκτικό, εκτός εάν είναι ισόμορφο με το $K_2$. Έστω κύκλος $C$ που περιέχει το $u$, αλλά όχι το $v$. Αυτός σίγουρα υπάρχει, διότι το 
$G'$ είναι συνεκτικό, οπότε παίρνοντας δύο ακμές της $u$ προς το $G'$, μαζί με το μονοπάτι μεταξύ των δύο αντίστοιχων κορυφών στο $G'$, ο κύκλος που σχηματίζεται δεν περνάει από το $v$. Όπως έχουμε
αποδείξει στο Λήμμα 1 της άσκησης 3.10, υπάρχουν 2 εσωτερικώς διακεκριμένα μονοπάτια από το $v$ σε δύο διαφορετικές κορυφές $x$ και $y$ του κύκλου $C$. Επίσης, επειδή ο $u$, $v$ είναι διαχωριστής, 
υπάρχει μονοπάτι
από την $u$ στην $v$ που δεν περνάει από καμία κορυφή του $G'$. Έχουμε λοιπόν τις κορυφές $u$, $v$, $x$, $y$ και ένα σύνολο μονοπατιών που συνδέουν κάθε ζευγάρι αυτών (τα ζευγάρια ($u$, $x$),
($u$,$y$) και ($x$,$y$) συνδέονται με μονοπάτια πάνω στον κύκλο), έτσι ώστε όλα τα μονοπάτια να
είναι ανά δύο εσωτερικώς διακεκριμένα. Αν συνθλίψουμε τις ακμές σε αυτά τα μονοπάτια, αφού πρώτα σβήσουμε τις ακμές που δεν ανήκουν στο μονοπάτι, καταλήγουμε στο $K_4$, άτοπο.

Τώρα έστω η αποσύνθεση του $H$ σε δισυνεκτικές συνιστώσες. Αν κάποια δισυνεκτική συνιστώσα διαφορετική από αυτές που περιέχουν τα $u$ και $v$ μοιράζεται κοινή κορυφή μόνο με μία άλλη δισυνεκτική
συνιστώσα, τότε σβήνοντας αυτή την κορυφή η δισυνεκτική συνιστώσα αποσυνδέεται από το υπόλοιπο γράφημα. Όμως, το $G$ γνωρίζουμε ότι είναι 2-συνεκτικό, άρα αυτό είναι άτοπο. Συνεπώς κάθε δισυνεκτική
συνιστώσα έχει κοινή κορυφή με τουλάχιστον δύο άλλες δισυνεκτικές συνιστώσες. Αν θεωρήσουμε ότι κάθε δισυνεκτική συνιστώσα είναι μία κορυφή και οι κοινές κορυφές δύο συνιστωσών είναι ακμές, τότε ο
μόνος τρόπος να μην δημιουργείται κύκλος είναι να έχουμε μονοπάτι από την κορυφή που αντιστοιχεί στο $u$ σε αυτήν που αντιστοιχεί στο $v$. Συνεπώς έχουμε μια αλυσίδα δισυνεκτικών συνιστωσών από το $u$ στο $v$, έστω $D_1$, $D_2$, ..., $D_k$, όπου $u\in D_1$, $v\in D_k$ και $V(D_i)\cup V(D_{i+1}) = v_i$. Κάθε ένα από τα $D_i$ είναι ένα δισυνεκτικό γράφημα που δεν περιέχει το $K_4$ ως ελασσον, αφού 
ούτε το $G$ το περιέχει. Συνεπώς όλα τα $D_i$ είναι σειριακά-παράλληλα.

Η παραπάνω ανάλυση ισχύει για κάθε συνεκτική συνιστώσα που ορίζει ο διαχωριστής $u$, $v$. Ξεκινάμε από το $K_2$, όπου οι κορυφές είναι οι $u$ και $v$. Διπλασιάζουμε την ακμή τόσες φορές, όσες είναι
και οι συνεκτικές συνιστώσες που ορίζει ο διαχωριστής. Τώρα, για κάθε συνεκτική συνιστώσα, υποδιαιρούμε την αντίστοιχη ακμή τόσες φορές, όσες είναι και οι αντίστοιχες δισυνεκτικές συνιστώσες (που όπως είπαμε
παραπάνω, αποτελούν αλυσίδα). Τώρα, σε κάθε ακμή αντιστοιχεί ένα σειριακό-παράλληλο γράφημα. Εφαρμόζουμε το μετασχηματισμό της ακμής αυτής στο ανίστοιχο 
σειριακό-παράλληλο γράφημα και καταλήγουμε στο $G$, άρα το $G$ είναι 
σειριακό-παράλληλο γράφημα, το οποίο είναι άτοπο. Άρα ισχύει το ζητούμενο.
	\end{proof}     
Λήμμα 1: Για κάθε γράφημα $G$ με $n(G)\geq 4$, ισχύει ότι $\kappa(G)\geq 3 \Rightarrow K_4 \subseteq_{\epsilon\lambda} G$.
	\begin{proof}
Είναι το Πόρισμα 5.44 από τις σημειώσεις του μαθήματος.
	\end{proof}
 
\end{enumerate}

\section{Χρωματισμοί και άλλα}
\begin{enumerate}

\item[6.7 ($\star$)]
Έστω $G$ τριμερές $(n+1)$-κανονικό γράφημα όπου κάθε μέρος του έχει $n$ κορυφές. Δείξτε ότι $K_3 \leq_{\text{υπ}} G$.

\begin{proof}
Έστω $S_1$, $S_2$, $S_3$ τα τρία σύνολα των κορυφών και έστω ότι δεν υπάρχει τρίγωνο. Θεωρούμε την κορυφή $v$ με το μέγιστο αριθμό γειτόνων σε ακριβώς ένα σύνολο $S_i$ και έστω $k$ αυτός ο αριθμός γειτόνων.
Επίσης, χωρίς βλάβη της γενικότητας $v\in S_1$ και οι $m$ γείτονές της βρίσκονται στο $S_2$. Τώρα, επειδή το γράφημα είναι $(n+1)$-κανονικό, η $v$ συνδέεται με κάποια κορυφή $u$ του $S_3$. Επειδή έχουμε
υποθέσει ότι δεν υπάρχει τρίγωνο, η $u$ μπορεί να συνδέεται το πολύ με $n-m$ κορυφές του $S_2$, άρα με τουλάχιστον $n+1-(n-m)=m+1$ κορυφές του $S_1$. Αυτό είναι άτοπο, αφού έχουμε υποθέσει ότι η $v$ έχει
το μέγιστο αριθμό ακμών προς κάποιο $S_i$. Άρα υπάρχει τρίγωνο, δηλαδή $K_3 \leq_{\text{υπ}} G$.
\end{proof}

\item[6.9 ($\star\star$)]
   Ένα γράφημα λέγεται άρτιο αν όλες οι κορυφές έχουν άρτιο βαθμό. Δείξτε ότι αν το G είναι συνεκτικό γράφημα, τότε 
   $|\{H\subseteq_{\text{πα}} G | H \text{είναι άρτιο}\}| = 2^{m(G)-n(G)+1}$.

   \begin{proof}
Θεωρούμε $S = \{H\subseteq_{\pi \alpha} G | H \text{άρτιο}\}$.
     Θα ορίσουμε μία 1-1 και επί συνάρτηση $f$ από το σύνολο 
$A = \{H | H\subseteq_{\pi\alpha} G\}$, δηλαδή το σύνολο των παραγόμενων γραφημάτων 
του $G$, στο $B = S\times \{X\subseteq V(G) | |X| mod 2 = 0\}$, 
δηλαδή το καρτεσιανό γινόμενο του συνόλου των άρτιων παραγόμενων γραφημάτων 
με την οικογένεια υποσυνόλων του $V(G)$ με άρτιο πληθάριθμο. Το σύνολο των 
παραγόμενων γραφημάτων του $G$ έχει πληθάριθμο $2^{m(G)}$, αφού κάθε ακμή 
μπορεί να υπάρχει ή να μην υπάρχει στο παραγόμενο υπογράφημα. 
Επίσης η οικογένεια υποσυνόλων του $V(G)$ με άρτιο πληθάριθμο έχει πληθάριθμο 
$2^{n(G)-1}$, αφού έχουμε 2 επιλογές για κάθε κορυφή (θα μπει ή δεν θα μπει 
στο υποσύνολο), εκτός από την τελευταία, της οποίας η τοποθέτηση καθορίζεται 
μοναδικά από το αν το υποσύνολο έχει άρτιο ή περιττό αριθμό κορυφών. Λόγω 
του λήμματος 1, η $f$ είναι 1-1 και επί, άρα έχουμε ότι 
$2^{m(G)} = |S| \cdot 2^{n(G)-1} \Rightarrow |S| = 2^{m(G)-n(G)+1}$, το οποίο 
είναι το ζητούμενο.
   \end{proof}


Λήμμα 1: Υπάρχει 1-1 και επί συνάρτηση από το σύνολο $A$ στο σύνολο $B$.
\begin{proof}
Για κάθε ζευγάρι κορυφών $i$, $j$ με $i\neq j$, ορίζουμε $P_{ij}$ ένα μονοπάτι μεταξύ τους στο $G$. Αυτό προφανώς υπάρχει, αφού το $G$ είναι συνεκτικό.
Ορισμός $f$: Έστω $Z\in A$ και $T$ το σύνολο των κορυφών του $Z$ με περιττό βαθμό. Είναι γνωστό ότι $|Z|\text{mod} 2=0$. Διαμερίζουμε τις κορυφές του $Z$ σε ζευγάρια
$(a_i, b_i)$ (με κάποιο μονοσήμαντο τρόπο, πχ αριθμούμε τις κορυφές του $Z$ $u_1,u_2,...,u_k$ και βάζουμε τα ζευγάρια $(u_1,u_2), ..., (u_{k-1},u_k)$)
και για κάθε ζευγάρι θεωρούμε το μονοπάτι $P_{a_i b_i}$ (το οποίο επίσης είναι μονοσήμαντο εκ κατασκευής). 
Για κάθε ακμή πάνω σε αυτό το μονοπάτι, αν υπάρχει στο $Z$ τότε την αφαιρούμε, ενώ αν δεν
υπάρχει την προσθέτουμε. Είναι εύκολο να δούμε ότι αυτός ο μετασχηματισμός διατηρεί την αρτιότητα των βαθμών των ενδιάμεσων κόμβων, και επίσης πλέον οι $a_i$, $b_i$
έχουν άρτιο βαθμό. Κάνοντας αυτό το μετασχηματισμό για κάθε ζευγάρι, θα καταλήξουμε με ένα άρτιο γράφημα $U$. Ορίζουμε $f(Z)=U\times T$. Ουσιαστικά η $f$ μετασχηματίζει ένα
γράφημα σε άρτιο, αλλά επιστρέφει και την πληροφορία του ποιοι κόμβοι ήταν περιττοί. Αντίστροφα, αν έχουμε ένα άρτιο γράφημα $U$ και ένα υποσύνολο $T$ του $V(G)$ με άρτιο πληθάριθμο,
θεωρούμε τη διαμέριση του $T$ σε ζευγάρια και για κάθε ζευγάρι εφαρμόζουμε τον ίδιο μετασχηματισμό που ορίσαμε παραπάνω. Έτσι θα πάρουμε ξανά το γράφημα $Z$ με $f(Z)=U\times T$.
Συνεπώς η $f$ είναι 1-1 και επί.
\end{proof}

\item[6.10 ($\star\star$)]
	Δείξτε ότι υπάρχει θετική σταθερά $c$, τέτοια ώστε αν για κάποιο γράφημα $G$ ισχύει ότι $\delta (G)\geq k$, τότε το $G$ περιέχει $c\cdot k^2$ ακμοδιακεκριμένους κύκλους.

\begin{proof}
Έστω $\delta(G)\geq k\geq 4$. Λόγω του λήμματος 2, έχουμε $\geq \lfloor \frac{k-1}{3}\rfloor$ (κορυφο-)διακεκριμένους κύκλους. Διαγράφουμε τις ακμές όλων αυτών των κύκλων. Στο γράφημα
$G'$ που θα προκύψει έχουμε $\delta(G')\geq k-2$. Εφαρμόζουμε επαναληπτικά την ίδια διαδικασία, έως ότου το γράφημα που απόμένει έχει $\delta(G')<4$. Συνολικά αυτή η διαδικασία θα επαναληφθεί
τουλάχιστον $\lfloor\frac{k}{2}\rfloor -1$ φορές. Οι ακμοδιακεκριμένοι κύκλοι που θα έχουμε συνολικά λοιπόν θα είναι τουλάχιστον $\lfloor \frac{k-1}{3}\rfloor + \lfloor \frac{k-3}{3}\rfloor + ... + 1 + 0 = \Theta (k^2)$.
\end{proof}

Λήμμα 1: Αν $\delta(G)\geq 4$, υπάρχει κύκλος με μήκος $\leq 2\cdot log_2 n$.
\begin{proof}
Έχουμε $m\geq \frac{\delta(G)\cdot n}{2}\geq 2n$, άρα η πυκνότητα είναι τουλάχιστον 2. Αυτό που μένει έχει αποδειχθεί στην άσκηση 1.10.
\end{proof}

Λήμμα 2: Σε κάθε γράφημα $G$ με $\delta(G)\geq k\geq 4$ υπάρχουν τουλάχιστον $\lfloor \frac{k-1}{3}\rfloor$ διακεκριμένοι κύκλοι.
\begin{proof}
Έστω ένας ελάχιστος κύκλος $C$. Αυτος λόγω του λήμματος 1 θα έχει μήκος το πολύ $2\cdot log_2 n$. Επίσης καμία κορυφή $u\in G-C$ δεν μπορεί να έχει πάνω από 3 ακμές προς κορυφές του $G$.
Αν είχε, τότε έστω δύο από αυτές και οι αντίστοιχες κορυφές του κύκλου. Αυτές θα είχαν απόσταση $\leq \lfloor \frac{|C|}{2}\rfloor$ στον $C$, άρα χρησιμοποιώντας αυτές τις δύο ακμές, 
θα υπήρχε κύκλος με μέγεθος το πολύ
$\lfloor\frac{|C|}{2}\rfloor + 2$, το οποίο για $|C|\geq 5$ είναι άτοπο αφού δημιουργεί κύκλο μικρότερο από τον ελάχιστο. Για $|C|=3$, είναι προφανές ότι δεν μπορούμε να έχουμε πάνω από 3
ακμές από κάποια κορυφή προς τις κορυφές του $C$, ενώ για $|C|=4$ αν είχαμε 4 ακμές προς κορυφές το $C$, θα σχηματιζόταν κύκλος μήκους 3, άτοπο.
Από το παραπάνω συμπεραίνουμε ότι το εναγόμενο γράφημα $G'$ του $G$ με σύνολο κορυφών το $G-C$ θα έχει $\delta(G') \geq k-3$. Εφαρμόζοντας επαναληπτικά την ίδια διαδικασία στο εναγόμενο 
γράφημα, μέχρι ο ελάχιστος
βαθμός του αντίστοιχου εναγόμενου γραφήματος να γίνει μικρότερος από 4, έχουμε συνολικά τουλάχιστον $\lfloor \frac{k-1}{3}\rfloor$ διακεκριμένους κύκλους.
\end{proof}

\end{enumerate}

\end{document}
