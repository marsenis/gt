% --- To be compiled with XeLaTeX ---
% ---      Encoding: UTF-8        ---

\documentclass[a4paper, oneside, 11pt]{article}

%fontspec package provides a configurable interface for font selection, and allows complex font choices to be named and later reused. It's needed for XeLaTeX
\usepackage[cm-default]{fontspec}

% Unicode support
\usepackage{xunicode}
\usepackage{xltxtra}

% Default words and phrases in Greek (e.g. 'Περίληψη' instead of 'Abstract'). Also contains hyphenation rules for Greek Language
\usepackage{xgreek}

% Mathematical fonts, theorems etc.
\usepackage{amsfonts}
\usepackage{amsmath}
\usepackage{amsthm}

% Default page layout for consuming a larger portion of the page.
\usepackage{fullpage}

% Greek fonts (Computer Modern)
\setmainfont[Mapping=tex-text]{CMU Serif}

% Auxiliary commands
\newcommand{\HRule}{\rule{\linewidth}{0.5mm}}

\begin{document}

\begin{titlepage}
\begin{center}

\includegraphics[width=0.3\textwidth]{./pyrforos.png}
\includegraphics[width=0.2\textwidth]{./uoa.png}\\[1cm]

\textsc{\LARGE Σχολή Ηλεκτρολόγων Μηχανικών και Μηχανικών Υπολογιστών}\\[1.5cm]

\HRule \\[0.4cm]
{\huge \bfseries Γραφοθεωρία\\
\LARGE Ομάδα Ασκήσεων No. 2}\\[0.4cm]

\HRule \\[1.5cm]

\begin{center}
\textbf{Ομάδα 7}\\
Αξιώτης Κυριάκος\\
Αρσένης Γεράσιμος
\end{center}

\vfill

{\large \today}
\end{center}

\end{titlepage}


\section{Βαθμοί, κύκλοι, μονοπάτια}

\begin{enumerate}
\item[1.9 ($\star$)]
   Για κάθε θετικό ακέραιο $\alpha$ και για κάθε γράφημα $G$, το $V(G)$
   περιέχει περισσότερες από $\left( 1-\frac1\alpha \right) \cdot n(G)$
   κορυφές βαθμού αυστηρά μικρότερου του $2\alpha\delta^*(G)$.

   \begin{proof}
      μπλα μπλα μπλα...
   \end{proof}

\item[1.10 ($\star\star$)]
   Κάθε γράφημα $G$ με τουλάχιστον 2 κορυφές και
   $\epsilon(G) \geq 2$, έχει περιφέρεια το πολύ $2 \cdot \log_2 (n)$.

   \begin{proof}
      μπλα μπλα..
   \end{proof}
\end{enumerate}

\section{Άκυκλα γραφήματα}
\section{Συνεκτικότητα}
\section{Εμβαπτίσεις}
\section{Δομές σε γραφήματα}
\section{Χρωματισμοί και άλλα}

\end{document}
