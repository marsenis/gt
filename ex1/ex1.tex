% --- To be compiled with XeLaTeX ---
% ---      Encoding: UTF-8        ---

\documentclass[a4paper, oneside, 11pt]{article}

%fontspec package provides a configurable interface for font selection, and allows complex font choices to be named and later reused. It's needed for XeLaTeX
\usepackage[cm-default]{fontspec}

% Unicode support
\usepackage{xunicode}
\usepackage{xltxtra}

% Default words and phrases in Greek (e.g. 'Περίληψη' instead of 'Abstract'). Also contains hyphenation rules for Greek Language
\usepackage{xgreek}

% Mathematical fonts, theorems etc.
\usepackage{amsfonts}
\usepackage{amsmath}
\usepackage{amsthm}

% Default page layout for consuming a larger portion of the page.
\usepackage{fullpage}

% Greek fonts (Computer Modern)
\setmainfont[Mapping=tex-text]{CMU Serif}

% Auxiliary commands
\newcommand{\HRule}{\rule{\linewidth}{0.5mm}}

\begin{document}

\begin{titlepage}
\begin{center}

\includegraphics[width=0.3\textwidth]{./pyrforos.png}
\includegraphics[width=0.2\textwidth]{./uoa.png}\\[1cm]

\textsc{\LARGE Σχολή Ηλεκτρολόγων Μηχανικών και Μηχανικών Υπολογιστών}\\[1.5cm]

\HRule \\[0.4cm]
{\huge \bfseries Γραφοθεωρία\\
\LARGE Ομάδα Ασκήσεων No. 2}\\[0.4cm]

\HRule \\[1.5cm]

\begin{center}
\textbf{Ομάδα 7}\\
Αξιώτης Κυριάκος\\
Αρσένης Γεράσιμος
\end{center}

\vfill

{\large \today}
\end{center}

\end{titlepage}


\section{Βαθμοί, κύκλοι, μονοπάτια}

\begin{enumerate}
\item[1.9 ($\star$)]
   Για κάθε θετικό ακέραιο $\alpha$ και για κάθε γράφημα $G$, το $V(G)$
   περιέχει περισσότερες από $\left( 1-\frac1\alpha \right) \cdot n(G)$
   κορυφές βαθμού αυστηρά μικρότερου του $2\alpha\delta^*(G)$.

   \begin{proof}
      Σύμφωνα με τον ορισμό έχουμε ότι
      $\delta^*(G) = \max \{ k\ |\ \exists H \subseteq G \text{ με }
       \delta(H) \geq k \}$.

      Θέλουμε να δείξουμε ότι:

      \[
         | \{ u\ |\ u \in V(G) \land d(u) < 2\alpha\delta^*(G) \} |
            > \left( 1 - \frac1\alpha \right) n(G)\\
      \]

      Έστω λοιπόν, προς απαγωγή σε άτοπο ότι:
      \begin{align*}
         | \{ u\ |\ u \in V(G) \land d(u) < 2\alpha\delta^*(G) \} |
            &\leq \left( 1 - \frac1\alpha \right) n(G)\\
         \Leftrightarrow
         n(G) - | \{ u\ |\ u \in V(G) \land d(u) \geq 2\alpha\delta^*(G) \} |
            &\leq \left( 1 - \frac1\alpha \right) n(G)\\ 
         \Leftrightarrow
         | \{ u\ |\ u \in V(G) \land d(u) \geq 2\alpha\delta^*(G) \} |
            &\geq \frac1\alpha n(G)\\
      \end{align*}

      Ισχύει όμως:

      \[ 2m(G) = \sum_{u \in V(G)} d(u) \geq
         \sum_{u \in V(G): d(u) \geq 2\alpha\delta*(G)} d(u)
         \geq \frac1\alpha n(G) \cdot 2\alpha\delta*(G)
         = 2n(G)\delta^*(G) \]

      \[ m(G) \geq n(G) \cdot \delta^*(G) \Leftrightarrow
         \epsilon(G) \geq \delta^*(G) \]

      Από το Πόρισμα 3.1 των σημειώσεων του μαθήματος γνωρίζουμε ότι
      $\delta^*(G) \geq \epsilon(G)$, συνεπώς θα έχουμε:

      \[ \epsilon(G) = \delta^*(G) \]

      TODO: Εδώ καταλήγουμε σε άτοπο γιατί αν ισχύει ισότητα τότε οι
      παραπάνω ανισότητες είναι αυστηρές και αυτό δεν μπορεί να ισχύει
      $[ \ldots ]$

   \end{proof}

\item[1.10 ($\star\star$)]
   Κάθε γράφημα $G$ με τουλάχιστον 2 κορυφές και
   $\epsilon(G) \geq 2$, έχει περιφέρεια το πολύ $2 \cdot \log_2 (n)$.

   \begin{proof}
      μπλα μπλα..
   \end{proof}
\end{enumerate}

\section{Άκυκλα γραφήματα}

\begin{enumerate}
   \item[2.10 $(\star)$]
      Σε κάθε δέντρο με $n$ κορυφές και διάμετρο τουλάχιστον
      $2k - 3$ υπάρχουν τουλάχιστον $n-k$ διαφορετικά μονοπάτια
      μήκους $k$.

      \begin{proof}

      *** ΛΑΘΟΣ ***

      Έστω $u, v$ δύο αντιδιαμετρικοί κόμβοι με $d(u, v) = \text{diam}(u, v)$
      και έστω $P$ το μονοπάτι που τους ενώνει. Υπάρχει κόμβος $w$ πάνω στο
      $P$ τέτοιος ώστε είτε $d(u, w) \geq k-1$ είτε $d(w, v) \geq k-1$
      (διαφορετικά θα είχαμε $\text{diam}(G) = d(u, v) = d(u, w) + d(w, v) \leq
      2(k-2) = 2k - 4$).

      Υποθέτουμε λοιπόν χωρίς βλάβη της γενικότητας ότι $d(u, w) \geq k-1$
      και μάλιστα επιλέγουμε το κοντινότερο τέτοιο $w$ στο $u$, δηλαδή
      $d(u, w) = k-1$.
      
      Στο μονοπάτι $P'$ από $u$ στον $w$ υπάρχουν ακριβώς $k$ κορυφές.
      Θα δείξουμε ότι από κάθε μία από τις υπόλοιπες $n-k$ κορυφές
      μπορούμε να δημιουργήσουμε μονοπάτια μήκους $k$ που να καταλήγουν
      σε κορυφές του $P'$ και τα μονοπάτια αυτά θα είναι διαφορετικά
      μεταξύ τους αφού το καθένα έχει διαφορετική αφετηρία.

      \end{proof}
\end{enumerate}

\section{Συνεκτικότητα}
\begin{enumerate}
   \item[3.10 $(\star\star)$]
Για κάθε $k$ κορυφές ενός $k$-συνεκτικού γραφήματος, υπάρχει κύκλος που να τις περιέχει όλες.
\begin{proof}
Έστω $k$ κορυφές του γραφήματος $G$ και $C$ κύκλος που περιέχει όσο το δυνατόν περισσότερες από τις $k$ κορυφές. Έστω $S$ το σύνολο των $k$ κορυφών. Αν $|C|=k$, τελειώσαμε. 
Διαφορετικά, έστω $|C|=l$ και $u$ μία από τις 
$k$ κορυφές, η οποία δεν 
βρίσκεται στον κύκλο. Από το Λήμμα 1, υπάρχουν $min(|C|,k)$ εσωτερικώς διακεκριμένα μονοπάτια από το $u$ προς τις κορυφές του κύκλου, και κανένα δεν τελειώνει στην ίδια κορυφή του κύκλου. Έστω $v_i$ μία
απαρίθμηση των κορυφών του κύκλου (με τη σειρά που εμφανίζονται πάνω στον κύκλο) οι οποίες αποτελούν άκρο κάποιου μονοπατιού από τα παραπάνω 
και $P_i$ τα αντίστοιχα μονοπάτια. Επίσης έστω $F_i$ το μονοπάτι από 
την $v_i$ στην $v_{i+1}$ το οποίο δεν περιέχει καμία άλλη από τις $v_j$. (Έχουμε θεωρήσει ότι $v_{min(|C|,k)+1}\equiv v_1$). Αν ο κύκλος έχει μήκος $l$, τότε περιέχει μόνο κορυφές από το $S$. Ο κύκλος
$v_1, P_1, u, P_2, v_2, v_3, ..., v_l, v_1$ περιέχει $l+1$ στοιχεία του $S$, άτοπο. Αν έχει μήκος $>l$, τότε οι κορυφές $v_i$ είναι τουλάχιστον $l+1$. Αυτό σημαίνει ότι υπάρχουν τουλάχιστον $l+1$
διαφορετικά μονοπάτια $F_i$. Άρα θα υπάρχει ένα $F_i$ το οποίο δεν περιέχει στο εσωτερικό του καμία κορυφή του $S$. Τότε, ο κύκλος $v_1, F_1, v_2, ..., v_i, P_i, u, P_{i+1}, v_{i+1}, F_{i+1}, ...,v_1$ 
έχει $l+1$ στοιχεία του $S$, άτοπο.
Άρα για κάθε σύνολο $k$ κορυφών, υπάρχει κύκλος που τις περιέχέι όλες.
\end{proof}

Λήμμα 1: Έστω $k$-συνεκτικό γράφημα, κύκλος του με τουλάχιστον $l$ κορυφές με $l<k$ και τυχαία κορυφή $u$ εκτός του κύκλου. Τότε υπάρχουν $l$ κορυφές του κύκλου $v_1, v_2, ..., v_l$ και 
εσωτερικώς διακεκριμένα μονοπάτια $P_i = u...v_i$ για κάθε $1\leq i\leq l$.
\begin{proof}
Έστω μία νέα κορυφή $v$ που συνδέεται με ακμή με όλες τις κορυφές του κύκλου. Δηλαδή θεωρούμε γράφημα $G$ με $V(G')=V(G)\cup \{v\}$ και $E(G')=E(G)\cup \{(v,x)|x\in C\}$. Το $G$ είναι $l$-συνεκτικό:
Αν σβήσουμε $l-1$ κορυφές και σε αυτές περιέχεται η $v$, τότε οι κορυφές που απομένουν συνδέονται λόγω της $k$-συνεκτικότητας του αρχικού γραφήματος. Σε διαφορετική περίπτωση, θα σβηστούν το πολύ $l-1$
κορυφές του κύκλου και συνεπώς θα μείνει τουλάχιστον μία άκμή από την $v$ προς μια κορυφή του κύκλου, άρα το γράφημα θα παράμείνει συνεκτικό. Αφού το γράφημα είναι $l$-συνεκτικό, θα υπάρχουν $l$ εσωτερικώς
διακεκριμένα μονοπάτια από την κορυφή $u$ στην κορυφή $v$. Κάθε ένα από αυτά τα μονοπάτια περνάει από τουλάχιστον μία κορυφή του κύκλου. Για κάθε μονοπάτι $P = u...v$, θεωρούμε την πρώτη φορά που περνάει
από μία κορυφή του κύκλου. Έστω ότι αυτή είναι η $x_i$. Το σύνολο των μονοπατιών $\{P_i = u...x_i\}$ είναι το ζητούμενο, αφού τα μονοπάτια είναι εσωτερικώς διακεκριμένα και καταλήγουν σε $l$ διαφορετικές
κορυφές του κύκλου.

\end{proof}

\end{enumerate}
\section{Εμβαπτίσεις}
\section{Δομές σε γραφήματα}

\begin{enumerate}
   \item[5.9 $(\star)$]
      Κάθε γράφημα περιέχει τουλάχιστον $\frac{m(G)(4m(G) - n^2(G))}{3n(G)}$
      τρίγωνα.

      \begin{proof}

      Έστω μια ακμή $\{u, v\}$. Η ιδέα είναι να βρούμε το ελάχιστο πλήθος
      τριγώνων στα οποία μπορεί να ανήκει αυτή η ακμή και έτσι μετά αθροίζοντας
      κατάλληλα να μπορέσουμε να φράξουμε από κάτω το συνολικό πλήθος των
      τριγώνων του γραφήματος.

      Ορίζουμε $U = N_G(u) \backslash v,
      V = N_G(v) \backslash u$. Ισχύει ότι $|U| + |V| = d(u) + d(v) - 2$.
      Επίσης, $|U \cup V| \leq n(G) - 2$ αφού δεν υπάρχουν πάνω από
      $n(G)-2$ κορυφές που να απομένουν στο γράφημα.

      Άρα, έχουμε:

      \[ |U \cap V| = |U| + |V| - |U \cup V| \geq d(u) + d(v) - n(G) \]

      Κάθε κορυφή που ανήκει στο $U \cap V$ δημιουργεί τρίγωνο με τις
      κορυφές $u, v$. Άρα το πλήθος των τριγώνων $|T_{\{u, v\}}|$ που μπορεί
      να ανήκει
      η ακμή $\{u, v\}$ είναι τουλάχιστον $d(u) + d(v) - n(G)$.

      Αν συμβολίσουμε με $T$ το σύνολο των τριγώνων του $G$ έχουμε:

      \[ 3|T| = \sum_{\{u, v\} \in E(G)} T_{\{u, v\}} \]

      επειδή κάθε τρίγωνο περιέχει 3 ακμές.

      Συνεπώς:

      \begin{align*}
         |T| &\geq \frac13 \sum_{\{u, v\} \in E(G)} ( d(u) + d(v) - n(G) )\\
             &= \frac13 \sum_{\{u, v\} \in E(G)} ( d(u) + d(v) ) -
                \frac{n(G)m(G)}3 \\
             &= \frac13 \sum_{u \in V(G)} d^2(u) - \frac{n(G)m(G)}3\\
             &\geq \frac1{3n(G)} \left( \sum_{u \in V(G)} d(u) \right)^2
                 - \frac{n(G)m(G)}3\\
             &= \frac{4m^2(G)}{3n(G)} - \frac{n(G)m(G)}3\\
             &= \frac{m(G)(4m(G) - n^2(G))}{3n(G)}\\
      \end{align*}

      Όπου το 4ο βήμα προκύπτει από την ανισότητα Cauchy-Schwarz:

      \[ d(u_1)\cdot1 + d(u_2)\cdot2 + \ldots + d(u_n)\cdot1
         \leq (d^2(u_1) + \ldots + d^2(u_n)) \cdot (1 + \ldots + 1)
         = (d^2(u_1) + \ldots + d^2(u_n)) \cdot n \]

      \end{proof}

   \item[5.10 $(\star \star)$]
THELEI FTIAKSIMO
Δείξτε ότι ένα πολυγράφημα είναι σειριακό-παράλληλο αν είναι 2-συνεκτικό και δεν περιέχει καμία υποδιαίρεση του $K_4$ ως ελάσσον. Ένα γράφημα καλείται σειριακό-παράλληλο αν μπορεί
να προκύψει από το $K_2$ μετά από σειρά υποδιαιρέσεων ακμών ή διπλασιασμών ακμών (δηλαδή αντικατάσταση μιας ακμής από μια διπλής πολλαπλότητας με τα ίδια άκρα).
	\begin{proof}
Αρχικά, αν ένα γράφημα περιέχει κάποια υποδιαίρεση του $K_4$ ως ελάσσον, τότε περιέχει και το $K_4$ ως ελάσσον, αφού η διάλυση κορυφής είναι η αντίστροφη πράξη της υποδιαίρεσης ακμής και γνωρίζουμε
ότι η σύνθλιψη ακμής μπορεί να προσομοιώσει την διάλυση κορυφής.
Θα δείξουμε κάτι πιο ισχυρό, δηλαδή ότι αν ένα πολυγράφημα είναι συνεκτικό και δεν περιέχει το $K_4$ ως ελάσσον, τότε είναι σειριακό-παράλληλο.
Για 2 κορυφές ισχύει, αφού έχουμε το $K_2$ που είναι σειριακό-παράλληλο.
Για 3 κορυφές επίσης ισχύει. Αν έχουμε το $P_3$, τότε προκύπτει από το $K_2$ με μία υποδιαίρεση ακμής. Αν έχουμε το $K_3$, αυτό μπορεί να προκύψει από την εξής ακολουθία κινήσεων: 
$K_2$->διπλασιασμός ακμής, υποδιαίρεση της μίας ακμής.
Θεωρούμε το γράφημα $G$ με τον ελάχιστο αριθμό κορυφών, το οποίο είναι συνεκτικό και δεν περιέχει το $K_4$ ως ελάσσον και δεν είναι σειριακό-παράλληλο. Από το Λήμμα 1, το γράφημα $G$ δεν μπορεί
να είναι 3-συνεκτικό. Αν το $G$ δεν είναι 2-συνεκτικό, τότε έχει κορυφή τομής. Έστω
Έστω ένας 2-διαχωριστής $u$, $v$ και $G'$ μία συνεκτική συνιστώσα που προκύπτει μετά τη διαγραφή των κορυφών $u$, $v$. Έστω γράφημα $H$ με 
$V(H)=V(G')\cup {u,v}$ και $E(H) = {(x,y) | x\in V(H), y\in V(H), (x,y)\in G, (x,y)\neq (u,v)}$. Το γράφημα $H$ είναι συνεκτικό, διότι σε διαφορετική περίπτωση κάποια από τις κορυφές 
$u$, $v$ θα αποτελούσε κορυφή τομής. Αν το $H$ δεν είναι 2-συνεκτικό, τότε λόγω του Λήμματος 2 είναι και σειριακό-παράλληλο με άκρα τα $u$, $v$. Αν είναι 2-συνεκτικό αλλά ότι 3-συνεκτικό, τότε 
λόγω του Λήμματος 3 
και της υπόθεσης ελαχιστότητας είναι σειριακό-παράλληλο με άκρα τα $u$, $v$. Λόγω του Λήμματος 3, το $H$ δεν μπορεί να είναι 3 συνεκτικό. Θεωρούμε λοιπόν το γράφημα $Κ_2$. Έστω $x$ το πλήθος των
συνεκτικών συνιστωσών που προκύπτουν στο $G$ μετά τη διαγραφή των $u$, $v$. Αν στο $G$ υπάρχει η ακμή $(u,v)$, τότε διπλασιάζουμε $x$ φορές την ακμή του $K_2$. Διαφορετικά τη διπλασιάζουμε $x-1$ φορές.
Τώρα, εφόσον έχουμε αποδείξει ότι κάθε συνιστώσα, μαζί με τις κορυφές $u$,$v$ είναι σειριακό-παράλληλο γράφημα με άκρα τα $u$, $v$, μπορεί να προκύψει με μια σειρά διπλασιασμών και υποδιαιρέσεων ακμών
από το $K_2$, δηλαδή από μία από τις ακμές που προέκυψαν από το διπλασιασμό. Αν επαναλάβουμε αυτή τη διαδικασία για κάθε συνιστώσα, θα προκύψει το γράφημα $G$. Αυτό σημαίνει ότι το $G$ είναι 
σειριακό-παράλληλο, το οποίο είναι άτοπο. Άρα η υπόθεση δεν ισχύει και κάθε 2-συνεκτικό γράφημα που δεν περιέχει το $K_4$ ως ελάσσον είναι σειριακό-παράλληλο. Παρατήρηση: Η συνθήκη ότι ένα γράφημα
δεν περιέχει το $K_4$ ως ελάσσον, με την προϋπόθεση ότι είναι συνεκτικό, είναι ικανή για την απόδειξη του ότι είναι σειριακό-παράλληλο.
	\end{proof}     
Λήμμα 1: Για κάθε γράφημα $G$ με $n(G)\geq 4$, ισχύει ότι $\kappa(G)\geq 3 \Rightarrow K_4 \subseteq_{\epsilon\lambda} G$.
	\begin{proof}
Είναι το Πόρισμα 5.44 από τις σημειώσεις του μαθήματος.
	\end{proof}
Λήμμα 2: Κάθε συνεκτικό γράφημα που δεν είναι 2-συνεκτικό είναι 
	\begin{proof}

	\end{proof}
 
\end{enumerate}

\section{Χρωματισμοί και άλλα}
\begin{enumerate}

\item[6.9 ($\star\star$)]
   Ένα γράφημα λέγεται άρτιο αν όλες οι κορυφές έχουν άρτιο βαθμό. Δείξτε ότι αν το G είναι συνεκτικό γράφημα, τότε 
   $|\{H\subseteq_{\text{πα}} G | H \text{είναι άρτιο}\}| = 2^{m(G)-n(G)+1}$.

   \begin{proof}
Θεωρούμε $S = \{H\subseteq_{\pi \alpha} G | H \text{άρτιο}\}$.
     Θα ορίσουμε μία 1-1 και επί συνάρτηση $f$ από το σύνολο 
$A = \{H | H\subseteq_{\pi\alpha} G\}$, δηλαδή το σύνολο των παραγόμενων γραφημάτων 
του $G$, στο $B = S\times \{X\subseteq V(G) | |X| mod 2 = 0\}$, 
δηλαδή το καρτεσιανό γινόμενο του συνόλου των άρτιων παραγόμενων γραφημάτων 
με την οικογένεια υποσυνόλων του $V(G)$ με άρτιο πληθάριθμο. Το σύνολο των 
παραγόμενων γραφημάτων του $G$ έχει πληθάριθμο $2^{m(G)}$, αφού κάθε ακμή 
μπορεί να υπάρχει ή να μην υπάρχει στο παραγόμενο υπογράφημα. 
Επίσης η οικογένεια υποσυνόλων του $V(G)$ με άρτιο πληθάριθμο έχει πληθάριθμο 
$2^{n(G)-1}$, αφού έχουμε 2 επιλογές για κάθε κορυφή (θα μπει ή δεν θα μπει 
στο υποσύνολο), εκτός από την τελευταία, της οποίας η τοποθέτηση καθορίζεται 
μοναδικά από το αν το υποσύνολο έχει άρτιο ή περιττό αριθμό κορυφών. Λόγω 
του λήμματος 1, η $f$ είναι 1-1 και επί, άρα έχουμε ότι 
$2^{m(G)} = |S| \cdot 2^{n(G)-1} \Rightarrow |S| = 2^{m(G)-n(G)+1}$, το οποίο 
είναι το ζητούμενο.
   \end{proof}


Λήμμα 1: Υπάρχει 1-1 και επί συνάρτηση από το σύνολο $A$ στο σύνολο $B$.
\begin{proof}
Για κάθε ζευγάρι κορυφών $i$, $j$ με $i\neq j$, ορίζουμε $P_{ij}$ ένα μονοπάτι μεταξύ τους στο $G$. Αυτό προφανώς υπάρχει, αφού το $G$ είναι συνεκτικό.
Ορισμός $f$: Έστω $Z\in A$ και $T$ το σύνολο των κορυφών του $Z$ με περιττό βαθμό. Είναι γνωστό ότι $|Z|\text{mod} 2=0$. Διαμερίζουμε τις κορυφές του $Z$ σε ζευγάρια
$(a_i, b_i)$ (με κάποιο μονοσήμαντο τρόπο, πχ αριθμούμε τις κορυφές του $Z$ $u_1,u_2,...,u_k$ και βάζουμε τα ζευγάρια $(u_1,u_2), ..., (u_{k-1},u_k)$)
και για κάθε ζευγάρι θεωρούμε το μονοπάτι $P_{a_i b_i}$. Για κάθε ακμή πάνω σε αυτό το μονοπάτι, αν υπάρχει στο $Z$ τότε την αφαιρούμε, ενώ αν δεν
υπάρχει την προσθέτουμε. Είναι εύκολο να δούμε ότι αυτός ο μετασχηματισμός διατηρεί την αρτιότητα των βαθμών των ενδιάμεσων κόμβων, και επίσης πλέον οι $a_i$, $b_i$
έχουν άρτιο βαθμό. Κάνοντας αυτό το μετασχηματισμό για κάθε ζευγάρι, θα καταλήξουμε με ένα άρτιο γράφημα $U$. Ορίζουμε $f(Z)=U\times T$. Ουσιαστικά η $f$ μετασχηματίζει ένα
γράφημα σε άρτιο, αλλά επιστρέφει και την πληροφορία του ποιοι κόμβοι ήταν περιττοί. Αντίστροφα, αν έχουμε ένα άρτιο γράφημα $U$ και ένα υποσύνολο $T$ του $V(G)$ με άρτιο πληθάριθμο,
θεωρούμε τη διαμέριση του $T$ σε ζευγάρια και για κάθε ζευγάρι εφαρμόζουμε τον ίδιο μετασχηματισμό που ορίσαμε παραπάνω. Έτσι θα πάρουμε ξανά το γράφημα $Z$ με $f(Z)=U\times T$.
Συνεπώς η $f$ είναι 1-1 και επί.
\end{proof}

\item[6.10 ($\star\star$)]
	Δείξτε ότι υπάρχει θετική σταθερά $c$, τέτοια ώστε αν για κάποιο γράφημα $G$ ισχύει ότι $\delta (G)\geq k$, τότε το $G$ περιέχει $c\cdot k^2$ ακμοδιακεκριμένους κύκλους.

\begin{proof}
Έστω $\delta(G)\geq k\geq 4$. Λόγω του λήμματος 2, έχουμε $\geq \lfloor \frac{k-1}{3}\rfloor$ εσωτερικώς διακεκριμένους κύκλους. Διαγράφουμε τις ακμές όλων αυτών των κύκλων. Στο γράφημα
$G'$ που θα προκύψει έχουμε $\delta(G)\geq k-2$. Εφαρμόζουμε επαναληπτικά την ίδια διαδικασία, έως ότου το γράφημα που απόμένει έχει $\delta(G')<4$. Συνολικά αυτή η διαδικασία θα επαναληφθεί
τουλάχιστον $\lfloor\frac{k}{2}\rfloor -1$ φορές. Οι ακμοδιακεκριμένοι κύκλοι που θα έχουμε συνολικά λοιπόν θα είναι τουλάχιστον $\lfloor \frac{k-1}{3}\rfloor + \lfloor \frac{k-3}{3}\rfloor + ... + 1 + 0 = \Theta (k^2)$.
\end{proof}

Λήμμα 1: Αν $\delta(G)\geq 4$, υπάρχει κύκλος με μήκος $\leq 2\cdot log_2 n$.
\begin{proof}
Έχουμε $m\geq \frac{\delta(G)\cdot n}{2}\geq 2n$, άρα η πυκνότητα είναι τουλάχιστον 2. Αυτό που μένει έχει αποδειχθεί στην άσκηση 1.10.
\end{proof}

Λήμμα 2: Σε κάθε γράφημα $G$ με $\delta(G)\geq k\geq 4$ υπάρχουν τουλάχιστον $\lfloor \frac{k-1}{3}\rfloor$ εσωτερικώς διακεκριμένοι κύκλοι.
\begin{proof}
Έστω ένας ελάχιστος κύκλος $C$. Αυτος λόγω του λήμματος 1 θα έχει μήκος το πολύ $2\cdot log_2 n$. Επίσης καμία κορυφή $u\in G-C$ δεν μπορεί να έχει πάνω από 3 ακμές προς κορυφές του $G$.
Αν είχε, τότε έστω δύο από αυτές και οι αντίστοιχες κορυφές του κύκλου. Αυτές θα είχαν απόσταση $\leq \lfloor \frac{|C|}{2}\rfloor$ στον $C$, άρα χρησιμοποιώντας αυτές τις δύο ακμές, 
θα υπήρχε κύκλος με μέγεθος το πολύ
$\lfloor\frac{|C|}{2}\rfloor + 2$, το οποίο για $|C|\geq 5$ είναι άτοπο αφού δημιουργεί κύκλο μικρότερο από τον ελάχιστο. Για $|C|=3$, είναι προφανές ότι δεν μπορούμε να έχουμε πάνω από 3
ακμές από κάποια κορυφή προς τις κορυφές του $C$, ενώ για $|C|=4$ αν είχαμε 4 ακμές προς κορυφές το $C$, θα σχηματιζόταν κύκλος μήκους 3, άτοπο.
Από το παραπάνω συμπεραίνουμε ότι το εναγόμενο γράφημα $G'$ του $G$ με σύνολο κορυφών το $G-C$ θα έχει $\delta(G') \geq k-3$. Εφαρμόζοντας επαναληπτικά την ίδια διαδικασία στο εναγόμενο 
γράφημα, μέχρι ο ελάχιστος
βαθμός του αντίστοιχου εναγόμενου γραφήματος να γίνει μικρότερος από 4, έχουμε συνολικά τουλάχιστον $\lfloor \frac{k-1}{3}\rfloor$ εσωτερικώς διακεκριμένους κύκλους.
\end{proof}

\end{enumerate}

\end{document}
