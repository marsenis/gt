% --- To be compiled with XeLaTeX ---
% ---      Encoding: UTF-8        ---

\documentclass[a4paper, oneside, 11pt]{article}

%fontspec package provides a configurable interface for font selection, and allows complex font choices to be named and later reused. It's needed for XeLaTeX
\usepackage[cm-default]{fontspec}

% Unicode support
\usepackage{xunicode}
\usepackage{xltxtra}

% Default words and phrases in Greek (e.g. 'Περίληψη' instead of 'Abstract'). Also contains hyphenation rules for Greek Language
\usepackage{xgreek}

% Mathematical fonts, theorems etc.
\usepackage{amsfonts}
\usepackage{amsmath}
\usepackage{amsthm}

% Default page layout for consuming a larger portion of the page.
\usepackage{fullpage}

% Greek fonts (Computer Modern)
\setmainfont[Mapping=tex-text]{CMU Serif}

% Auxiliary commands
\newcommand{\HRule}{\rule{\linewidth}{0.5mm}}

\newtheorem{thm}{Θεώρημα}
\newtheorem{lm}[thm]{Λήμμα}
\newtheorem{obsrv}[thm]{Παρατήρηση}

\theoremstyle{definition}
\newtheorem{defn}[thm]{Ορισμός}

\begin{document}

\begin{titlepage}
\begin{center}

\includegraphics[width=0.3\textwidth]{./pyrforos.png}
\includegraphics[width=0.2\textwidth]{./uoa.png}\\[1cm]

\textsc{\LARGE Σχολή Ηλεκτρολόγων Μηχανικών και Μηχανικών Υπολογιστών}\\[1.5cm]

\HRule \\[0.4cm]
{\huge \bfseries Γραφοθεωρία\\
\LARGE Ομάδα Ασκήσεων No. 2}\\[0.4cm]

\HRule \\[1.5cm]

\begin{center}
\textbf{Ομάδα 7}\\
Αξιώτης Κυριάκος\\
Αρσένης Γεράσιμος
\end{center}

\vfill

{\large \today}
\end{center}

\end{titlepage}


\section{Χρωματισμοί κορυφών και ακμών}
\begin{enumerate}
   \item[1.6] \emph{Έστω $G$ γράφημα όπου $\Delta(G) \leq 3$. Δείξτε ότι το
              $G$ είναι 4-ακμοχρωματίσιμο.}

   Θα δείξουμε ότι γραμμικό γράφημα $L(G)$ του $G$ είναι 4 χρωματίσιμο.

   Από το Θεώρημα Brooks έχουμε ότι το $L(G)$ θα είναι
   $\Delta(L(G))$-χρωματίσιμο εκτός αν είναι περιττός κύκλος
   ή κλίκα όπου σε αυτή την περίπτωση θα είναι $(\Delta(L(G)) + 1)$-χρωματίσιμο.

   Αν το $L(G)$ είναι περιττός κύκλος τότε θα είναι 3-χρωματίσιμο.

   Αν είναι κλίκα με 3 ή λιγότερες κορυφές τότε προφανώς είναι
   3-χρωματίσιμο ενώ αν είναι κλίκα με τουλάχιστον 4 κορυφές τότε
   περιέχει το $K_4$ ως υπογράφημα όμως αυτό δεν γίνεται
   σύμφωνα με το Λήμμα \ref{lm1.6.1}.

   Επομένως το $L(G)$ θα είναι $\Delta(L(G))$-χρωματίσιμο και από
   την Παρατήρηση \ref{lm1.6.2} συμπεραίνουμε ότι θα είναι
   4-χρωματίσιμο.
   
   \begin{lm}
      \label{lm1.6.1}
      Αν $K_4 \subseteq L(G)$ τότε $\Delta(G) \geq 4$.
   \end{lm}
   \begin{proof}
      Έστω $e_1, e_2, e_3, e_4$ οι ακμές του $G$ που στο $L(G)$ είναι
      κορυφές 4-κλίκας. Αυτό σημαίνει ότι κάθε ζεύγος $e_i, e_j$ θα πρέπει
      να έχει κοινό άκρο.

      Έστω $e_1 = \{u, v\}$ και χωρίς βλάβη της γενικότητας έστω
      $e_2 = \{u, w\}$. Αν η $e_3$ έχει κοινό άκρο με την $e_1$
      την κορυφή $v$, τότε αναγκαστικά $e_3 = \{v, w\}$ ώστε να έχει
      κοινό άκρο και με την $e_3$. Σε αυτή την περίπτωση όμως η $e_4$
      δεν μπορεί να έχει κοινό άκρο και με τις 3 προηγούμενες ακμές.

      Άρα η $e_3$ έχει κοινό άκρο με την $e_1$ το $u$, δηλαδή
      $e_3 = \{ u, x \}$ για κάποια κορυφή $x$ (διαφορετική από
      τις $\{u, v, w\}$).

      Τέλος, η $e_4$ θα πρέπει να έχει κοινό άκρο με όλες τις υπόλοιπες
      και αυτό μπορεί να συμβεί μόνο αν $e_4 = \{u, y\}$ για κάποια
      νέα κορυφή $y$.

      Συνεπώς $\Delta(G) \geq d(u) = 4$.
   \end{proof}

   \begin{obsrv}
      \label{lm1.6.2}
      Αν $\Delta(G) \leq 3$ τότε $\Delta(L(G)) \leq 4$.
   \end{obsrv}
   \begin{proof}
      Έστω ότι υπήρχε ακμή $e = \{u, v\} \in E(G)$ η οποία να έχει κοινό
      άκρο με τουλάχιστον 5 άλλες ακμές στο $G$. Αυτό σημαίνει ότι
      σε ένα από τα άκρα της $e$, έστω στο $u$, θα προσπίπτουν τουλάχιστον
      3 από αυτές τις 5 ακμες και έτσι η $u$ θα έχει βαθμό τουλάχιστον
      4 το οποίο είναι άτοπο.
   \end{proof}

   \item[1.7] \emph{Δείξτε ότι υπάρχει $c$ τέτοιο ώστε κάθε ένωση δύο επίπεδων
              γραφημάτων να έχει χρωματικό αριθμό το πολύ $c$.}
   
   \begin{lm}
      \label{lm1.7.1}
      Αν $G = G_1 \cup G_2$ τότε $\chi(G) \leq \chi(G_1) \cdot \chi(G_2)$.
   \end{lm}
   \begin{proof}
      Έστω $\chi(G_1) = k, \chi(G_2) = l$
      και $\chi_{G_1} : V(G_1) \rightarrow [k],
      \chi_{G_2} : V(G_2) \rightarrow [l]$ οι συναρτήσεις
      χρωματισμού του καθενός.
      
      Επεκτείνουμε τις παραπάνω συναρτήσεις ως εξής:
      \[ \overline{\chi_{G_i}}(u) = \left\{
         \begin{array}{cc}
            \chi_{G_i}(u) & ,\ u \in V(G_i)\\
            1 & , \text{ διαφορετικά}
         \end{array}
         \right.
      \]

      Ορίζουμε το σύνολο $S = \{ (x, y)\ |\ x \in A, y \in B \}$ και
      χρωματίζουμε το $G$ με χρώματα από το $S$ ως εξής:

      \[ \chi_G(u) = \left( \overline{\chi_{G_1}}(u),
                     \overline{\chi_{G_2}}(u) \right) \]

      Ο παραπάνω είναι έγκυρος χρωματισμός αφού αν $\chi_G(u) = \chi_G(v)$
      τότε $\overline{\chi_{G_i}}(u) = \overline{\chi_{G_i}}(v)$
      για $i = 1, 2$ επομένως $\{ u, v \} \notin E(G_i)$ και έτσι
      $\{u, v\} \notin E(G)$.

      Άρα $\chi(G) \leq |S| = \chi(G_1) \cdot \chi(G_2)$.
   \end{proof}

   Από το θεώρημα των 4 χρωμάτων έχουμε ότι αν $G_1, G_2$ επίπεδα γραφήματα
   τότε $\chi(G_1), \chi(G_2) \leq 4$ επομένως από το Λήμμα \ref{lm1.7.1}:
   $\chi(G_1 \cup G_2) \leq 16$.
\end{enumerate}

\section{Διαπεράσεις}
\begin{enumerate}
   \item[2.1] \emph{$(\star)$ Για ποιά $k$ και $l$ το γράφημα
              $G_{k, l} = P_l^{[k]}$ είναι Χαμιλτονιανό;}

   Για $k = 1$, κανένα από τα $P_l$ με $l \geq 1$ δεν είναι Χαμιλτονιανό.

   Για $k \geq 2$, θα δείξουμε ότι το $P_l^{[k]}$ είναι Χαμιλτονιανό
   αν και μόνο αν το $l$ είναι άρτιος.

   \begin{obsrv}
      Το $P_l^{[k]}$ είναι ένα $k$-διάστατο πλέγμα (grid). Στο εξής
      θα αριθμούμε τις κορυφές του με βάση τις συντεταγμένες
      στις οποίες βρίσκονται, δηλαδή:

      \[ V(P_l^{[k]}) = \{ (x_1, \ldots, x_k)\ |\  1 \leq x_i \leq l \text{ για }
      1 \leq i \leq k \} \]

      Για τις ακμές έχουμε:

      \[ E(P_l^{[k]}) = \{ \{ (x_1, \ldots, x_k), (y_1, \ldots, y_k) \}
         \ |\ \exists i:\ (|x_i - y_i| = 1
         \land \forall j \neq i:\ x_j = y_j) \} \]
   \end{obsrv}

   \begin{obsrv}
      \label{lm2.1.1}
      Το $P_l^{[2]}$ για $l$ άρτιο είναι Χαμιλτονιανό.
   \end{obsrv}
   \begin{proof}
      Ξεκινάμε από την πάνω αριστερά κορυφή και διατρέχουμε τις
      κορυφές της πρώτης στήλης προς τα κάτω. Έπειτα διατρέχουμε
      από κάτω προς τα πάνω τις κορυφές της δεύτερης στήλης
      μέχρι τη γραμμή 2 και συνεχίζουμε έτσι μέχρι να διατρέξουμε
      από κάτω προς τα πάνω (επειδή το $l$ είναι άρτιο) τις κορυφές
      τις τελευταίας στήλης όπου και κλείνουμε τον κύκλο διατρέχοντας
      στο τέλος τις κορυφές της πρώτης γραμμής.
   \end{proof}

   \begin{lm}
      \label{lm2.1.2}
      Αν $G$ είναι Χαμιλτονιανό τότε το $G \times P_k$ είναι επίσης
      Χαμιλτονιανό.
   \end{lm}
   \begin{proof}
      Το γράφημα $G \times P_k$ είναι ουσιαστικά το $G$ όπου κάθε κορυφή του
      έχει αντικατασταθεί από ένα μονοπάτι $P_k$ (και έχουν προστεθεί οι
      κατάλληλες ακμές μεταξύ κορυφών των μονοπατιών).

      Ας πάρουμε ένα κύκλο Hamilton του $G$:
      
      \[ u_1 \rightarrow \ldots \rightarrow u_n \rightarrow u_1 \]

      Αυτός μπορεί να μετασχηματιστεί απευθείας
      σε κύκλο Hamilton του $G \times P_k$ ως εξής:
      
      \[ (u_1^1 \rightarrow \ldots \rightarrow u_1^k) \rightarrow
         \ldots \rightarrow (u_n^1 \rightarrow \ldots
         \rightarrow u_n^k) \rightarrow u_1^1 \]

      όπου στο παραπάνω $u_i^j$ είναι η $j$-οστή κορυφή του μονοπατιού
      το οποίο έχει αντικαταστήσει την κορυφή $u_i$ του $G$ στον
      $G \times P_k$.
   \end{proof}

   Από το Λήμμα \ref{lm2.1.2} και την Παρατήρηση \ref{lm2.1.1} έχουμε
   επαγωγικά ότι για κάθε $k \geq 2$ και για $l$ άρτιο το $P_l^{[k]}$
   είναι Χαμιλτονιανό.

   \begin{lm}
      \label{lm2.1.3}
      Αν $l$ περιττός τότε το $G = P_l^{[k]}$ \emph{δεν} είναι Χαμιλτονιανό.
   \end{lm}
   \begin{proof}
      Ορίζουμε την εξής διαμέριση των κορυφών του $G$
      σε δύο σύνολα $A_1, A_2$:

      \[ A_i = \left\{ (x_1, \ldots, x_k) \in V(G)\ |\
               \sum_{i=j}^{k} x_j \equiv i \pmod{2} \right\}
               \text{ για } i = 1, 2\]

      Είναι φανερό ότι δεν υπάρχει ακμή μεταξύ κορυφών που βρίσκονται
      στο ίδιο σύνολο γιατί τα αθροίσματα των συντεταγμένων γειτονικών κορυφών
      διαφέρουν ακριβώς κατά ένα.

      Παρατηρούμε επίσης ότι το $G$ έχει $l^k$ κορυφές το οποίο είναι
      περιττός αριθμος για $l$ περιττό επομένως ένα από τα δύο
      σύνολα $A_1, A_2$ θα είναι μεγαλύτερο από το άλλο. Χωρίς βλάβη
      της γενικότητας υποθέτουμε $|A_1| > |A_2|$.

      Θεωρούμε λοιπόν το γράφημα $G \backslash A_2$ το οποίο
      θα πρέπει να έχει $|A_1| > |A_2|$ συνεκτικές συνιστώσες αποτελούμενες
      από μία κορυφή η κάθε μία. Όπως δείχνουμε όμως στην άσκηση
      2.10, ένα τέτοιο γράφημα δεν μπορεί να είναι Χαμιλτονιανό.
   \end{proof}


   \item[2.7] \emph{$(\star)$ Έστω $G$ συνεκτικό γράφημα τέτοιο ώστε το συμπλήρωμά του να είναι ατρίγωνο. Δείξτε ότι το $G$ έχει Χαμιλτονιανό μονοπάτι}

Έστω το μέγιστο μονοπάτι στο γράφημα. Αν όλες οι κορυφές είναι πάνω σε αυτό το μονοπάτι, έχουμε τελειώσει. Διαφορετικά, υπάρχει μια κορυφή $u$ που είναι έξω από το μονοπάτι. Η $u$ δεν μπορεί να έχει
ακμή προς κάποιο από τα δύο άκρα του μονοπατιού, αφού τότε θα είχαμε άτοπο στη μεγιστότητα του μονοπατιού. Επειδή όμως το συμπλήρωμα είναι ατρίγωνο, θα πρέπει να υπάρχει ακμή μεταξύ των δύο ακρών
του μονοπατιού, έχουμε δηλαδή έναν κύκλο $C$. Τώρα, επειδή το γράφημα είναι συνεκτικό, θα υπάρχει ακμή από κάποια κορυφή $v$ εκτός του κύκλου προς κάποια κορυφή του κύκλου.
Αυτό όμως σημαίνει ότι υπάρχει μεγαλύτερο μονοπάτι από αυτό που υποθέσαμε ως μέγιστο, άτοπο.
Άρα το $G$ έχει Χαμιλτονιανό μονοπάτι.

\includegraphics[width=0.3\textwidth]{./pics/graph3.png}

   \item[2.10] \emph{$(\star)$ Αν το γράφημα $G$ είναι Χαμιλτονιανό και $S\subseteq V(G)$, τότε το πλήθος των συνεκτικών συνιστωσών του $G\backslash S$ είναι το πολύ $|S|$.}

Έστω ότι το πλήθος των συνεκτικών συνιστωσών $c$ μπορεί να είναι μεγαλύτερο του $|S|$. Για κάθε συνεκτική συνιστώσα του $G\backslash S$,
οι ακμές που βγαίνουν στο αρχικό γράφημα από τις κορυφές της συνδέονται μόνο με το $S$. Για να υπάρχει κύκλος Hamilton, 
πρέπει να υπάρχουν τουλάχιστον 2 τέτοιες ακμές για κάθε συνιστώσα στον κύκλο Hamilton (σε διαφορετική περίπτωση θα είχαμε
γέφυρα). Σε κάθε ακμή από κάποια συνιστώσα του $G\backslash S$ προς το $S$ αντιστοιχεί και μια κορυφή του $S$ και επειδή στον κύκλο Hamilton όλες οι κορυφές έχουν βαθμό 2, κάθε κορυφή μπορεί να
αντιστοιχεί σε το πολύ 2 συνεκτικές συνιστώσες. Αυτό σημαίνει ότι $|S|\geq \frac{2\cdot c}{2} >  |S|$, άτοπο. Άρα ισχύει το ζητούμενο.


   \item[2.11] \emph{$(\star)$ Ένα τριγωνοποιημένο επίπεδο γράφημα έχει
               χρωματικό αριθμό 3 αν και μόνο αν είναι γράφημα Euler.}

   Θα θεωρήσουμε ότι το γράφημα περιέχει τουλάχιστον 3 κορυφές αφού
   διαφορετικά η πρόταση είναι τετριμμένη.

   Δείχνουμε τις δύο κατευθύνσεις της εκφώνησης ως εξής:

   \begin{itemize}
      \item[$(\Rightarrow)$]
         Έστω (προς απαγωγή σε άτοπο) ότι το $G$ (με $n(G) \geq 3$)
         τριγωνοποιημένο επίπεδο γράφημα το οποίο είναι 3-χρωματίσιμο
         αλλά \emph{δεν} είναι γράφημα Euler.

         Το $G$ θα πρέπει να περιέχει τουλάχιστον μία κορυφή περιττού
         βαθμού, έστω $u \in V(G)$. Η $u$ δεν μπορεί να έχει βαθμό 1
         γιατί διαφορετικά θα βρίσκεται στο σύνορο μίας μόνο
         όψης $f$ η οποία όμως θα πρέπει να έχει στο σύνορό της τουλάχιστον
         άλλες 2 κορυφές. Έστω $v, w$ αυτές οι κορυφές και χωρίς βλάβη της
         γενικότητας έστω $v$ η γειτονική της $u$. Τότε όμως μπορούμε
         να προσθέσουμε την ακμή $\{w, u\}$ και το γράφημα να παραμείνει
         επίπεδο. Αυτό είναι άτοπο γιατί το γράφημα είναι τριγωνοποιημένο,
         δηλαδή η προσθήκη μιας ακμής δεν θα έπρεπε να είναι εφικτή.

         Συνεπώς $d(u) \geq 3$. Έστω $[v_0, v_1, \ldots, v_{k-1}]$
         οι γειτονικές
         κορυφές τις $u$ σε ορολογιακή διάταξη όπως εμφανίζονται στην
         επίπεδη εμβάπτιση του $G$. Αφού το γράφημα είναι τριγωνοποιημένο
         θα πρέπει να υπάρχουν οι ακμές $\{v_i, v_{(i+1) \mod k}\}$ για κάθε
         $i = 0, \ldots, k-1$.

         Άρα η γειτονιά της $u$ ενάγει περιττό κύκλο και αυτό σημαίνει
         ότι χρειάζονται τουλάχιστον 4 χρώματα για το χρωματισμό
         της $u$ και της γειτονιάς της. Άτοπο.

      \item[$(\Leftarrow)$]
         Έστω τριγωνοποιημένο επίπεδο γράφημα $G$ με $n(G) \geq 3$
         το οποίο είναι γράφημα Euler αλλά \emph{δεν} είναι
         3-χρωματίσιμο.

         Από την εικασία του Hadwinger για $k=4$, έχουμε ότι
         $K_4 \leq G$, δηλαδή υπάρχει μια ακολουθία συνθλίψεων
         ακμών μετά από την οποία το γράφημα $G'$ που απομένει
         περιέχει 4-κλίκα.

         Κάθε κορυφή του $G$ έχει άρτιο βαθμό (ως γράφημα Euler) και
         έτσι το ίδιο θα ισχύει και για κάθε γράφημα που προκύπτει
         από συνθλίψεις ακμών του $G$. Συνεπώς το $G'$ θα είναι
         γράφημα Euler.

         Έστω $x, y, z, w$ οι κορυφές τις 4-κλίκας του $G'$.

         TODO: ... test
   \end{itemize}
\end{enumerate}

\section{Επίπεδα γραφήματα}
\section{Τέλεια γραφήματα}
\begin{enumerate}
	\item[4.3] \emph{$(\star)$ Δείξτε ότι για κάθε θετικό ακέραιο $k$ ισχύει ότι ένα γράφημα $G$ είναι τέλειο αν και μόνο αν το $G^{(k)}$ είναι τέλειο.}

Το $G^{(k)}$ είναι ουσιαστικά $k$ αντίγραφα του $G$, με επιπλέον ακμές ανάμεσα σε κάθε δύο κορυφές που βρίσκονται σε διαφορετικά αντίγραφα του $G$.
Από την ισχυρή εικασία των τέλειων γραφημάτων, αν το γράφημα $G$ δεν είναι τέλειο θα έχει περιττή τρύπα μεγέθους τουλάχιστον $5$. Άρα προφανώς και το $G^{(k)}$ θα έχει περιττή τρύπα μεγέθους τουλάχιστον
$5$, αφού περιέχει αντίγραφα του $G$ και κατά την ένωση προστίθενται ακμές μόνο μεταξύ διαφορετικών αντιγράφων του $G$. Άρα ούτε το $G^{(k)}$ θα είναι τέλειο.
Αντίστροφα, έστω ότι το $G^{(k)}$ δεν είναι τέλειο, δηλαδή έχει περιττή τρύπα μεγέθους τουλάχιστον $5$. Αν αυτή περιέχεται στο εσωτερικό ενός αντιγράφου του $G$, τότε περιέχεται και στο $G$ και άρα ούτε
το $G$ είναι τέλειο. Σε διαφορετική περίπτωση, αν οι κορυφές της τρύπας ανήκουν σε τουλάχιστον τρία διαφορετικά αντίγραφα του $G$, τότε θα σχηματίζεται τρίγωνο, το οποίο είναι άτοπο, αφού έχουμε τρύπα
μεγέθους τουλάχιστον $5$.
Η μόνη περίπτωση που μένει είναι οι κορυφές της τρύπας να ανήκουν σε ακριβώς δύο αντίγραφα του $G$, το οποίο είναι και αυτό άτοπο: Κάθε αντίγραφο μπορεί να περιέχει το πολύ δύο κορυφές της τρύπας, γιατί
διαφορετικά θα υπήρχε κορυφή με τρεις γείτονες στην τρύπα. Άρα η μόνη περίπτωση που μπορούμε να έχουμε τρύπα μεγέθους τουλάχιστον $5$ είναι αν αυτή υπάρχει στο $G$, δηλαδή ούτε το $G$ είναι τέλειο.

\end{enumerate}

\section{Μερικές διατάξεις}
\begin{enumerate}
	\item[5.5] \emph{$(\star)$ Δείξτε ότι για κάθε $k$, η κλάση των γραφημάτων με $vc(G)\leq k$ είναι καλώς μερικώς διατεταγμένη ως προς υπογραφήματα.}

Θα υποθέσουμε το αντίθετο, δηλαδή ότι υπάρχει άπειρη αντιαλυσίδα γραφημάτων με $vc(G)\leq k$, κανένα ζευγάρι από τα οποία δεν είναι υπογράφημα του άλλου. Τότε προφανώς
θα υπάρχει $k$, για το οποίο υπάρχει άπειρη αντιαλυσίδα με $vc(G)=k$. Έστω $S$ το σύνολο της κάλυψης ($|S|=k$) και $T=V(G) \backslash S$.
Επίσης επειδή οι διαφορετικοί συνδυασμοί ακμών μεταξύ των κορυφών του $S$ είναι $2^{{k\choose 2}}$, 
δηλαδή πεπερασμένοι, θα υπάρχει ένας από αυτούς, τον οποίο αν σταθεροποιήσουμε θα υπάρχει άπειρη αντιαλυσίδα. 
 Κάθε στοιχείο του $T$ συνδέεται με ένα υποσύνολο των στοιχείων
του $S$. Διαμερίζουμε το σύνολο $T$ με βάση με ποιο υποσύνολο του $S$ είναι συνδεδεμένο με ακμή. Αυτό διαμερίζει το $T$ σε $2^k-1$ σύνολα. Θα αναπαραστήσουμε τα πλήθη αυτών των συνόλων με ένα
σημείο στο $\mathbb{N}^{2^k-1}$. Συγκεκριμένα, η $i$-οστή συντεταγμένη αυτού του σημείου ισούται με το πλήθος των κόμβων που βρίσκονται στο $i$-οστό σύνολο της διαμέρισης. Αν όλες οι συντεταγμένες ενός 
σημείου είναι μικρότερες ή ίσες από αυτές ενός άλλου σημείου, τότε είναι εμφανές ότι το πρώτο γράφημα είναι υπογράφημα του δεύτερου. Αν ορίσουμε λοιπόν τη σχέση μερικής διάταξης
$(x_1,x_2,...,x_m) \leq (y_1,y_2,...,y_m) \Leftrightarrow \forall i\in [1,m] x_i \leq y_i$. Θα αποδείξουμε ότι δεν υπάρχει άπειρη αντιαλυσίδα ως προς αυτή τη σχέση, άρα η αρχική μας υπόθεση είναι άτοπη.

Για να το αποδείξουμε αυτό για κάθε διάσταση, θεωρούμε την ελάχιστη διάσταση $d$ για την οποία υπάρχει άπειρη αντιαλυσίδα. Δεν μπορεί να είναι $d=1$ αφού όλοι οι φυσικοί είναι συγκρίσιμοι μεταξύ τους. 
Έστω τώρα ότι $d>1$. Θεωρούμε ένα στοιχείο $(x_1,x_2,...,x_d)$ της αντιαλυσίδας. Για κάθε άλλο στοιχείο $(y_1,y_2,...,y_d)$, θα πρέπει να υπάρχει $i\in [1,d]$ έτσι ώστε $y_i<x_i$, διότι αλλιώς αυτά
τα δύο στοιχεία θα είναι συγκρίσιμα. Αφού η αντιαλυσίδα είναι άπειρη και οι διαστάσεις πεπερασμένες, θα υπάρχει $i\in [1,d]$ έτσι ώστε να υπάρχει άπειρη αντιαλυσίδα με $y_i<x_i$ για κάθε στοιχείο της
αλυσίδας.
Επειδή όμως το $x_i$ είναι πεπερασμένο, υπάρχουν πεπερασμένα τέτοια διαφορετικά $y_i$, και άρα θα υπάρχει άπειρη αντιαλυσίδα έτσι ώστε όλα τα στοιχεία της να έχουν την ίδια $i$-οστή συντεταγμένη.
Τότε, όμως, αν αγνοήσουμε την $i$-οστή συντεταγμένη, έχουμε βρει μια άπειρη αντιαλυσίδα στις $d-1$ διαστάσεις. Αυτό είναι άτοπο, αφού έχουμε θεωρήσει το $d$ ως ελάχιστο αντιπαράδειγμα.


	\item[5.6] \emph{$(\star)$ Δείξτε ότι, για κάθε $k$, κάθε γράφημα στο σύνολο παρεμπόδισης ελασσόνων της κλάσης $C_k=\{G|vc(G)\leq k\}$ έχει $O(k^2)$ κορυφές.}

Αρκεί να δείξουμε ότι κάθε γράφημα $G$ με $vc(G)>k$ έχει ως ελάσσον ένα $H$ με $vc(H)>k$ και $O(k^2)$ κορυφές. Στην πραγματικότητα θα δείξουμε ότι περιέχει σαν εναγόμενο υπογράφημα ένα τέτοιο γράφημα.
Έστω γράφημα $G$ με $vc(G)>k$ και έστω $S$ το σύνολο που πραγματοποιεί την κάλυψη ($|S|=vc(G)$). Επίσης θεωρούμε τη διαμέριση του $S$ σε δύο σύνολα $A$ και $B$, ώστε οι κορυφές του $A$ να 
είναι αυτές που έχουν τουλάχιστον
$k+1$ ακμές προς το $V(G)\backslash A$, και $B$ οι υπόλοιπες. Διακρίνουμε τρεις περιπτώσεις: 
\begin{itemize}

\item{a.}
Αν έχουμε ότι $|A| \geq k+1$, τότε διαγράφουμε οποιεσδήποτε $|A|-(k+1)$ κορυφές του $A$, όλες τις κορυφές του $B$, καθώς και όλες τις κορυφές του 
$V(G)\backslash S$ που έγιναν απομονωμένες. Στη συνέχεια, για κάθε κορυφή στο $A$, μαρκάρουμε οποιουσδήποτε $k+1$ γείτονες στο $V(G)\backslash S$. Αν κάποια κορυφή του $V(G)\backslash S$ δεν έχει
μαρκαριστεί, διαγράφεται και αυτή. Στο γράφημα $G'$ 
που έχει προκύψει, έχουμε κάλυψη με $k+1$ κορυφές χρησιμοποιώντας όλα τα στοιχεία του $A$. Αν δεν χρησιμοποιήσουμε έστω και ένα στοιχείο του $A$,
θα πρέπει να είναι στο σύνολο της κάλυψης οι $k+1$ γείτονες που έχει στο $V(G)\backslash S$. Άρα έχουμε $vc(G')>k$.

\item{b.}
Αν $|S|=\Theta(k)$, τότε μαρκάρουμε όλους τους γείτονες των κορυφών του $B$ στο $V(G)\backslash S$, αλλά και οποιουσδήποτε $k+1$ γείτονες στο $V(G)\backslash S$, για κάθε μία κορυφή του $A$.
Διαγράφουμε όλες τις κορυφές του $V(G)\backslash S$ που δεν μαρκάραμε.
Στο γράφημα $G'$ που προέκυψε, κάθε κορυφή του $S$ πλέον έχει $O(k)$ γείτονες στο $V(G)\backslash S$, άρα συνολικά έχουμε $O(k^2)$ κορυφές. Επίσης, από το ίδιο επιχείρημα που χρησιμοποιήσαμε στην
περίπτωση 1, όλες οι κορυφές του $A$ είναι στο σύνολο κάλυψης. Έστω τώρα ότι υπάρχει κάλυψη μικρότερη από $|S|$. Αυτό θα σήμαινε ότι το σύνολο $B$ θα μπορούσαμε στην αρχική κάλυψη να το
αντικαταστήσουμε με ένα μικρότερο $B'$. Αυτό γιατί καμία από τις κορυφές που σβήστηκαν από το $G$ δεν είχαν ακμή προς το $B$, συνεπώς οι ακμές τους ικανοποιούνταν από το $A$. Αυτό είναι όμως άτοπο,
άρα $vc(G')\geq\min(vc(G),k+1)>k$.

\item{c.}
Σε αυτή την περίπτωση έχουμε $|A|\leq k$ και $|S|=\omega(k)$ (άρα και $|B|=\omega(k)$). Τώρα, όπως και στα προηγούμενα, μαρκάρουμε για κάθε στοιχείο του $A$, οποιουσδήποτε $k+1$ γείτονές του στο
$V(G)\backslash S$. Στη συνέχεια διαλέγουμε οποιεσδήποτε $k+1$ κορυφές από το $B$, μαρκάρουμε όλους τους γείτονες κάθε μίας στο $V(G)\backslash S$ και σβήνουμε όλες τις υπόλοιπες κορυφές του $B$,
φτιάχνοντας έτσι ένα νέο σύνολο $B'$.
Τέλος, σβήνουμε όλες τις κορυφές του $V(G)\backslash S$ που δεν έχουν μαρκαριστεί ή είναι απομονωμένες. Είναι προφανές ότι έχουμε καταλήξει σε ένα γράφημα $G'$
με $O(k^2)$ κορυφές. Όλα τα στοιχεία του $A$ θα ανήκουν
στο σύνολο κάλυψης, και τα υπόλοιπα που θα ανήκουν στο σύνολο κάλυψης δεν μπορεί να είναι λιγότερα από $B'$, καθώς καμία από τις κορυφές του $V(G)\backslash S$ που έχουν σβηστεί δεν έχει ακμή
προς το $B'$. Συνεπώς έχουμε $vc(G')\geq |A| + |B'| > k$.

\end{itemize}
Σε κάθε περίπτωση, λοιπόν, ένα γράφημα $G$ με $vc(G)>k$ έχει εναγόμενο υπογράφημα $H$ με $vc(H)>k$ και $O(k^2)$ κορυφές, και άρα το ζητούμενο έχει αποδειχθεί.
\end{enumerate}
\section{$k$-δέντρα}

\begin{enumerate}
   \item[6.2] \emph{Καλούμε μερικό $k$-δέντρο κάθε υπογράφημα $k$-δέντρου.
   Δείξτε ότι το $K_{r,r}$ είναι μερικό $r$-δέντρο αλλά δεν είναι μερικό
   $(k-1)$-δέντρο.}

   Το $K_{r,r}$ είναι μερικό $k$-δέντρο αφού μπορούμε να το παράγουμε
   ως εξής:

   Ξεκινάμε με το $K_{r+1}$ και διαλέγουμε μία κορυφή του την οποία
   αναθέτουμε στο σύνολο $X$ και τις υπόλοιπες τις αναθέτουμε στο σύνολο
   $Y$. Το $Y$ είναι μια $r$-κλίκα επομένως μπορούμε να τοποθετήσουμε
   $r-1$ νέες κορυφές στο $X$ κάθε μία από τις οποίες τις συνδέουμε
   με όλες τις κορυφές του $Y$.

   Τώρα αφαιρούμε όλες τις ακμές μεταξύ κορυφών του $Y$ και αυτό
   που μένει είναι το $K_{r,r}$.

   Έστω τώρα ότι το $K_{r,r}$ ήταν μερικό $(r-1)$-δέντρο. Τότε θα πρέπει
   να περιέχει μια κορυφή $u$ με $d(u) < r$ (η τελευταία κορυφή που προσθέσαμε
   κατα της κατασκευή του $(r-1)$-δέντρου είχε βαθμό $r-1$). Αυτό όμως είναι
   άτοπο γιατί όλες οι κορυφές του $K_{r,r}$ έχουν βαθμό ίσο με $r$.


	\item[6.4] \emph{$(\star)$ Αν ένα χορδικό γράφημα είναι επίπεδο, τότε θα είναι και μερικό $3$-δέντρο.}

Γνωρίζουμε ότι ένα γράφημα έχει δενδροπλάτος $k$ αν και μόνο αν η μεγαλύτερη κλίκα της χορδικής κλειστότητάς του είναι $k+1$. Εφόσον έχουμε χορδικό γράφημα, αυτό ταυτίζεται με την χορδική του κλειστότητα,
και μάλιστα εφόσον είναι επίπεδο, δεν μπορεί να έχει κλίκα μεγαλύτερη του $4$. Αυτό σημαίνει ότι το δενδροπλάτος του είναι το πολύ $3$, δηλαδή θα είναι μερικό $3$-δέντρο.

	\item[6.5] \emph{Δείξτε ότι ο τρισδιάστατος υπερκύβος είναι μερικό $3$-δέντρο.}

Στο παρακάτω σχήμα απεικονίζεται μία χορδική κλειστότητα του τρισδιάστατου κύβου, η οποία εύκολα φαίνεται ότι είναι επίπεδο γράφημα. Συνεπώς, από την άσκηση $6.4$, ο τρισδιάστατος υπερκύβος είναι μερικό 
$3$-δέντρο.
\newline
\includegraphics[width=0.3\textwidth]{./pics/graph1.png}


\end{enumerate}

\section{Άπειρα γραφήματα}

\begin{enumerate}
   \item[7.3] \emph{$(\star)$ Χρησιμοποιώντας το λήμμα του K\H{o}nig, αποδείξτε
   ότι αν το $G$ είναι γράφημα όπου $|V(G)| = \aleph_0$ και κάθε υπογράφημά
   του είναι 3-χρωματίσιμο, τότε και το $G$ είναι 3-χρωματίσιμο.}

   Έστω $V(G) = \{ 1, 2, \ldots, n, \ldots \}$. Συμβολίζουμε με $G[k]$
   το εναγόμεμο υπογράφημα του $G$ με κορυφές τις $\{1, \ldots, k\}$.

   Δημιουργούμε το εξής δέντρο $T$: Κάθε κόμβος του δέντρου εκτός της ρίζας
   αντιστοιχεί σε ένα έγκυρο 3-χρωματισμό του $G[k]$ για κάποιο $k$.
   Συγκεκριμένα, η ρίζα έχει 3 παιδιά που αντιστοιχούν στους τρεις πιθανούς
   χρωματισμούς του $G[1]$ και αν ένας κόμβος $u \in T$ αντιστοιχεί σε
   3-χρωματισμό
   του $G[k]$, τότε θεωρούμε το γράφημα $G[k+1] \supseteq G[k]$ καθώς και κάθε
   3-χρωματισμό του που συμφωνεί με το χρωματισμό του $G[k]$. Υπάρχουν
   3 τέτοιοι χρωματισμοί (3 επιλογές για το χρώμα της νεας κορυφής).
   και ώς παιδία της $u$ θέτουμε τους έγκυρους από αυτούς
   τους χρωματισμούς.

   Παρατηρούμε ότι ένας κόμβος $u$ βρίσκεται σε απόσταση $r$ από τη ρίζα
   του $T$ αν και μόνο αν το $u$ αντιστοιχεί σε έγκυρο 3-χρωματισμό
   του $G[r]$.

   Για το γράφημα $T$ γνωρίζουμε ότι κάθε κόμβος έχει πεπερασμένο βαθμό
   (το πολύ 4) και ότι έχει άπειρο πλήθος κόμβων γιατί σύμφωνα με την
   προηγούμενη παρατήρηση, αν τo $G[k]$ είναι 3-χρωματίσιμο θα πρέπει
   να υπάρχει τουλάχιστον μια κορυφή $u$ που να αντιστοιχεί στο
   χρωματισμό του. Ξέρουμε όμως ότι όλα τα $G[k]$ για $k \in \mathbb{N}$
   είναι 3-χρωματίσιμα άρα θα πρέπει να υπάρχει τουλάχιστον μια κορυφή
   για κάθε τέτοιο $k$.

   Από το λήμμα του K\H{o}nig έχουμε λοιπόν ότι πρέπει να υπάρχει άπειρο
   μονοπάτι $P$ που να ξεκινάει από τη ρίζα. Το μονοπάτι αυτό ορίζει
   έναν 3-χρωματισμό του $G$ (το χρώμα μιας κορυφής $w \in V(G)$
   είναι το χρώμα που του αναθέτει ο χρωματισμός του $G[w]$ στο μονοπάτι
   $P$). O χρωματισμός αυτός είναι έγκυρος γιατί διαφορετικά, αν υπάρχουν
   κορυφές $u, v \in V(G)$ με $\{u, v\} \in E(G)$ και ίδιο χρώμα, τότε
   ο χρωματισμός του $G[\max(u, v)]$ στο μονοπάτι $P$ δεν θα ήταν έγκυρος.
\end{enumerate}

\section{Κανονικά γραφήματα και Ταιριάσματα}
\begin{enumerate}
	\item[8.2] \emph{$(\star\star)$ Δείξτε ότι κάθε συνεκτικό γράφημα με άρτιο αριθμό ακμών μπορεί να προσανατολιστεί έτσι ώστε κάθε κορυφή να έχει άρτιο εξώβαθμο.
Χρησιμοποιώντας αυτό δείξτε ότι κάθε $3$-κανονικό γράφημα με $4k$ κορυφές περιέχει ανεξάρτητο σύνολο με $k$ κορυφές το οποίο αν αφαιρεθεί από το $G$ δημιουργεί γράφημα
του οποίου όλες οι συνεκτικές συνιστώσες είναι μονοκυκλικές}

\begin{proof}
Αρχικά θα αποδείξουμε το πρώτο. Έστω ένας τυχαίος προσανατολισμός των ακμών του γραφήματος. Αυτός διαμερίζει τις κορυφές σε δύο σύνολα, το $A$ που περιέχει τις κορυφές με άρτιο
εξώβαθμο, και το $B$ που περιέχει τις κορυφές με περιττό εξώβαθμο. Αν $out_v$ είναι ο εξώβαθμος της κορυφής $v$ και $m$ το πλήθος των κορυφών του γραφήματος, 
γνωρίζουμε ότι $m=\sum_{v\in V} out_v = \sum_{v\in A} out_v + \sum_{v\in B} out_v$. Εφόσον το $m$ και ο πρώτος όρος του δεύτερου μέλους είναι άρτιοι, έχουμε ότι και ο δεύτερος όρος του δεύτερου
μέλους είναι άρτιος. Δεδομένου ότι για όλα τα $v\in B$ το $out_v$ είναι περιττό, θα πρέπει το $|B|$ να είναι άρτιο. Διαμερίζουμε τώρα το $B$ σε ζεύγη $(x_{2i-1},x_{2i})$ για $i\in [1,\frac{|B|}{2}]$.
Για κάθε ζεύγος βρίσκουμε ένα μονοπάτι (στο μη κατευθυνόμενο γράφημα) μεταξύ των $x_{2i-1}$ και $x_{2i}$ και για κάθε μία ακμή αυτού του μονοπατιού, αντιστρέφουμε την κατεύθυνσή της. Αυτό θα διατηρήσει
τον εξώβαθμο $mod 2$ όλων των κορυφών εκτός από τις $x_{2i-1}$ και $x_{2i}$, οι οποίες πλέον θα έχουν άρτιο εξώβαθμο. Κάνοντας την παραπάνω διαδικασία για όλα τα $\frac{|B|}{2}$ ζευγάρια, κάθε κορυφή
του γραφήματός μας έχει πλέον άρτιο εξώβαθμο.
\end{proof}

\begin{proof}
Στη συνέχεια εφαρμόζουμε στο γράφημά μας τον προσανατολισμό του παραπάνω Λήμματος, οπότε κάθε κορυφή έχει εξώβαθμο 0 ή 2. Στην πραγματικότητα, επειδή το άθροισμα των εξώβαθμων είναι ίσο με το
πλήθος των ακμών του γραφήματος και το τελευταίο είναι ίσο με $3\cdot 4k / 2=6k$, θα έχουμε ότι υπάρχουν ακριβώς $k$ κορυφές με εξώβαθμο $0$ και ακριβώς $3k$ κορυφές με εξώβαθμο $2$. Θεωρούμε
ως ανεξάρτητο σύνολο το σύνολο των κορυφών με εξώβαθμο $0$. Είναι προφανώς ανεξάρτητο, αφού αν υπήρχε ακμή μεταξύ αυτών των κορυφών, κάποια από τα δύο άκρα της θα είχε μη μηδενικό εξώβαθμο. Επιπλέον,
το πλήθος των ακμών που θα έχει το γράφημα μετά τη διαγραφή του ανεξάρτητου συνόλου είναι $6k-3k=3k$, αλλά και το πλήθος των κορυφών που θα μείνουν στο γράφημα είναι $4k-k=3k$. Αυτό σημαίνει ότι η
πυκνότητα του γραφήματος που απομένει είναι $1$. Αν αποδείξουμε ότι καμία συνεκτική συνιστώσα δεν μπορεί να είναι δέντρο, τότε κάθε συνιστώσα θα έχει πυκνότητα τουλάχιστον $1$, και άρα θα πρέπει
κάθε συνιστώσα να έχει πυκνότητα ακριβώς $1$, δηλαδή να είναι μονοκυκλική. Έστω τώρα ένας κόμβος $u$ σε μια συνεκτική συνιστώσα $S$. Αφού κάθε κορυφή που δεν ανήκει στο ανεξάρτητο σύνολο έχει εξώβαθμο
$2$, θα έχει και εσώβαθμο $1$. Ακολουθώντας από την $u$ τις προσανατολισμένες ακμές κατά την αντίθετη κατεύθυνση, φτιάχνουμε μια ακολουθία κορυφών με μη μηδενικό εξώβαθμο. Προφανώς αυτή η ακολουθία θα
είναι πεπερασμένη και δεν γίνεται να περιέχει κάποιο κόμβο του ανεξάρτητου συνόλου, αφού αυτοί έχουν μηδενικό εξώβαθμο. Αυτό σημαίνει ότι η ακολουθία θα αρχίσει να επαναλαμβάνεται, άρα θα υπάρχει κύκλος.
Συνεπώς κάθε συνεκτική συνιστώσα που προκύπτει μετά από τη διαγραφή του ανεξάρτητου συνόλου θα έχει πυκνότητα τουλάχιστον $1$ και το ζητούμενο έχει αποδειχθεί.
\end{proof}

	\item[8.4] \emph{$(\star)$ Βρείτε ένα γράφημα που να είναι ακμομεταβατικό αλλά όχι κορυφομεταβατικό και ένα γράφημα που να είναι κορυφομεταβατικό αλλά όχι ακμομεταβατικό.}

Παρακάτω παρουσιάζονται τα 2 γραφήματα:

\includegraphics[width=0.3\textwidth]{./pics/graph2.png}

Το πρώτο είναι ακμομεταβατικό, αλλά όχι κορυφομεταβατικό, αφού δεν είναι κανονικό.
\newline
Το δεύτερο είναι ουσιαστικά το τετράεδρο με κομμένες τις γωνίες, άρα εύκολα φαίνεται ότι είναι κορυφομεταβατικό. Δεν είναι όμως ακμομεταβατικό, αφού κάποιες ακμές ανήκουν σε τρίγωνο, ενώ άλλες όχι.


\end{enumerate}
\section{Διάφορα}

\begin{enumerate}
   \item[9.7] \emph{$(\star)$ Ποιά είναι η συνεκτικότητα του υπερκύβου
   $r$ διαστάσεων;}

   Θα δείξουμε με επαγωγή ότι $\kappa(Q_r) = r$.

   Για $r = 1$ το $Q_1$ περιέχει μόνο μία ακμή και είναι συνεκτικό.

   Αν ο $Q_{r-1}$ είναι $(r-1)$-συνεκτικός τότε θα δείξουμε ότι
   ο $Q_r = Q_{r-1} \times P_1$ είναι $r$-συνεκτικός.

   Ο $Q_r$ ως γνωστόν αποτελείται από δύο αντίγραφα $A_1, A_2$ του $Q_{r-1}$
   μαζί με τις ακμές που συνδέουν αντίστοιχες κορυφές μεταξύ τους.
   Στο εξής, αν έχουμε μια κορυφή $u \in V(A_1)$ θα συμβολίζουμε
   με $u'$ την κορυφή του $A_2$ με την οποία συνδέεται η $u$ στο
   $Q_r$.

   Θα δείξουμε ότι για οποιεσδήποτε δύο κορυφές $u, v \in V(Q_r)$
   υπάρχουν $r$ εσωτερικώς διακεκριμένα μονοπάτια από την $u$ στην
   $v$ διακρίνοντας τις εξής περιπτώσεις:

   \begin{itemize}
      \item $u, v \in V(A_1)$ (αντίστοιχα και για το $A_2$).

      Από την Ε.Υ. υπάρχουν $r-1$ εσωτερικώς διακεκριμένα μονοπάτια
      από την $u$ στην $v$ που χρησιμοποιούν μόνο ακμές μόνο από το
      $A_1$. Επίσης υπάρχει τουλάχιστον ένα μονοπάτι $P$ μεταξύ των
      $u'$ και $v'$ στο $A_2$ επομένως μπορούμε να δημιουργήσουμε
      το $P' = [u, u'] \cup P \cup [v', v]$ που δεν έχει κοινές κορυφές με
      τα υπόλοιπα $r-1$ εκτός από τα άκρα.

      \item $u \in V(A_1)$ και $v \in V(A_2)$ (ή αντίστροφα).

      Έστω $P_i$ για $i = 1, \ldots, r-1$ τα $r-1$ εσωτερικώς
      διακεκριμένα μονοπάτια μεταξύ των $u$ και $v$ στο $A_1$ και
      $P_i'$ τα αντίστοιχα μονοπάτια στο $A_2$. Συβολίζουμε με
      $x_i$ τον προτελευταίο κόμβο του μονοπατιού $P_i$.

      Με βάση τα μονοπάτια αυτά δημιουργούμε τα παρακάτω $r$
      εσωτερικώς διακεκριμένα μονοπάτια $R_i$:

      \begin{equation*}
         R_i = \left\{
            \begin{array}{cl}
               [u, u']\cup P_1' &, \ i = 1\\
               (P_i \backslash v) \cup [x_i, x_i', v'] &, \ i = 2, \ldots, r-1\\
               P_1 \cup [v, v'] &,\ i = r\\
            \end{array}
            \right.
      \end{equation*}

      %TODO: Να βάλουμε σχήμα
   \end{itemize}


   \item[9.9] \emph{$(\star)$ Κάθε $3$-συνεκτικό μη διμερές γράφημα έχει τουλάχιστον $4$ περιττούς κύκλους.}
\begin{proof}
	Εφόσον το γράφημα δεν είναι διμερές, θα έχει περιττό κύκλο. Έστω ο ελάχιστος περιττός κύκλος. Προφανώς αυτός δεν θα έχει χορδές, αφού έτσι θα υπήρχε ακόμα μικρότερος περιττός κύκλος (εφόσον
	κάθε χορδή χωρίζει τον κύκλο σε έναν άρτιο και έναν περιττό). Επιπλέον, θα υπάρχει κορυφή $u$ εξωτερική του κύκλου $C$, 
	αφού γνωρίζουμε ότι ο κύκλος δεν είναι $3$-συνεκτικό γράφημα. Από το Λήμμα 1, υπάρχουν 3 εσωτερικώς διακεκριμένα μονοπάτια από την $u$ σε διαφορετικές κορυφές του $C$. Έστω 
	$u\equiv P_1^i,P_2^i...,P_{k_i}^i$ για $i\in [1,3]$ αυτά τα τρία μονοπάτια. Για κάθε ζεύγος αυτών, σχηματίζονται δύο κύκλοι. Χωρίς βλάβη της γενικότητας για τα $1$ και $2$, ακολουθούμε το
	μονοπάτι $P^1$, κινούμαστε πάνω στον κύκλο προς την κορυφή $P_{k_2}^2$ (έχουμε δύο τρόπους να το κάνουμε αυτό) και στη συνέχεια ακολουθούμε το μονοπάτι $P^2$ ανάποδα. Καθώς οι δύο εναλλακτικές
	διαδρομές πάνω στον κύκλο τον καλύπτουν ολόκληρο, τα μήκη τους θα έχουν διαφορετικό 
	υπόλοιπο $mod 2$, άρα τουλάχιστον ένας από τους δύο κύκλους που ορίσαμε θα είναι περιττός.
	Αυτό σημαίνει ότι για κάθε ζευγάρι μονοπατιών $P^i$,$P^j$ έχουμε βρει έναν περιττό κύκλο. Αν σε αυτούς
	μετρήσουμε και τον $C$, έχουμε συνολικά βρει $4$ περιττούς κύκλους.
\end{proof}
Λήμμα 1:
\newline
	Έστω $k$-συνεκτικό γράφημα, κύκλος $C$ και κορυφή $u$ που δεν ανήκει στον κύκλο. Τότε υπάρχουν $\min(|C|,k)$ εσωτερικώς διακεκριμένα μονοπάτια από την $u$ προς διαφορετικές κορυφές του
	κύκλου $C$.
\begin{proof}
	Έχει αποδειχθεί στην πρώτη σειρά ασκήσεων.
\end{proof}
	
\end{enumerate}

\end{document}
