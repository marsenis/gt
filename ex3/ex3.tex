% --- To be compiled with XeLaTeX ---
% ---      Encoding: UTF-8        ---

\documentclass[a4paper, oneside, 11pt]{article}

%fontspec package provides a configurable interface for font selection, and allows complex font choices to be named and later reused. It's needed for XeLaTeX
\usepackage[cm-default]{fontspec}

% Unicode support
\usepackage{xunicode}
\usepackage{xltxtra}

% Default words and phrases in Greek (e.g. 'Περίληψη' instead of 'Abstract'). Also contains hyphenation rules for Greek Language
\usepackage{xgreek}

% Mathematical fonts, theorems etc.
\usepackage{amsfonts}
\usepackage{amsmath}
\usepackage{amsthm}

% Default page layout for consuming a larger portion of the page.
\usepackage{fullpage}

% Greek fonts (Computer Modern)
\setmainfont[Mapping=tex-text]{CMU Serif}

% Auxiliary commands
\newcommand{\HRule}{\rule{\linewidth}{0.5mm}}

\newtheorem{thm}{Θεώρημα}
\newtheorem{lm}[thm]{Λήμμα}
\newtheorem{obsrv}[thm]{Παρατήρηση}

\theoremstyle{definition}
\newtheorem{defn}[thm]{Ορισμός}

\begin{document}

\begin{titlepage}
\begin{center}

\includegraphics[width=0.3\textwidth]{./pyrforos.png}
\includegraphics[width=0.2\textwidth]{./uoa.png}\\[1cm]

\textsc{\LARGE Σχολή Ηλεκτρολόγων Μηχανικών και Μηχανικών Υπολογιστών}\\[1.5cm]

\HRule \\[0.4cm]
{\huge \bfseries Γραφοθεωρία\\
\LARGE Ομάδα Ασκήσεων No. 2}\\[0.4cm]

\HRule \\[1.5cm]

\begin{center}
\textbf{Ομάδα 7}\\
Αξιώτης Κυριάκος\\
Αρσένης Γεράσιμος
\end{center}

\vfill

{\large \today}
\end{center}

\end{titlepage}


\section{Χρωματισμοί κορυφών και ακμών}
\begin{enumerate}
   \item[1.6] \emph{Έστω $G$ γράφημα όπου $\Delta(G) \leq 3$. Δείξτε ότι το
              $G$ είναι 4-ακμοχρωματίσιμο.}

   Θα δείξουμε ότι γραμμικό γράφημα $L(G)$ του $G$ είναι 4 χρωματίσιμο.

   \begin{lm}
      \label{lm1.6.1}
      Αν $K_4 \subseteq L(G)$ τότε $\Delta(G) \geq 4$.
   \end{lm}
   \begin{proof}
      Έστω $e_1, e_2, e_3, e_4$ οι ακμές του $G$ που στο $L(G)$ είναι
      κορυφές 4-κλίκας. Αυτό σημαίνει ότι κάθε ζεύγος $e_i, e_j$ θα πρέπει
      να έχει κοινό άκρο.

      Έστω $e_1 = \{u, v\}$ και χωρίς βλάβη της γενικότητας έστω
      $e_2 = \{u, w\}$. Αν η $e_3$ έχει κοινό άκρο με την $e_1$
      την κορυφή $v$, τότε αναγκαστικά $e_3 = \{v, w\}$ ώστε να έχει
      κοινό άκρο και με την $e_3$. Σε αυτή την περίπτωση όμως η $e_4$
      δεν μπορεί να έχει κοινό άκρο και με τις 3 προηγούμενες ακμές.

      Άρα η $e_3$ έχει κοινό άκρο με την $e_1$ το $u$, δηλαδή
      $e_3 = \{ u, x \}$ για κάποια κορυφή $x$ (διαφορετική από
      τις $\{u, v, w\}$).

      Τέλος, η $e_4$ θα πρέπει να έχει κοινό άκρο με όλες τις υπόλοιπες
      και αυτό μπορεί να συμβεί μόνο αν $e_4 = \{u, y\}$ για κάποια
      νέα κορυφή $y$.

      Συνεπώς $\Delta(G) \geq d(u) = 4$.
   \end{proof}

   Εφόσον $\Delta(G) \leq 3$, από το Λήμμα \ref{lm1.6.1} έχουμε ότι
   το $L(G)$ δεν μπορεί να περιέχει το $K_4$ ως υπογράφημα άρα δεν μπορεί
   να το περιέχει και ως ελάσσον.

   Από την εικασία του Handwinger για την περίπτωση $k=4$ (για το συγκεκριμένο
   $k$ έχει αποδειχθεί ότι η εικασία ισχύει) έχουμε ότι $\chi(L(G)) < 4$
   άρα μπορούμε να χρωματίσουμε τις ακμές του $G$ με 4 (ή λιγότερα)
   χρώματα.

   % TODO: Check the proof again. Edeiksa kati pio isxyro opote mporei
   %       na exw kanei lathos.

   \item[1.7] \emph{Δείξτε ότι υπάρχει $c$ τέτοιο ώστε κάθε ένωση δύο επίπεδων
              γραφημάτων να έχει χρωματικό αριθμό το πολύ $c$.}
   
   \begin{lm}
      \label{lm1.7.1}
      Αν $G = G_1 \cup G_2$ τότε $\chi(G) \leq \chi(G_1) \cdot \chi(G_2)$.
   \end{lm}
   \begin{proof}
      Έστω $\chi(G_1) = k, \chi(G_2) = l$
      και $\chi_{G_1} : V(G_1) \rightarrow [k],
      \chi_{G_2} : V(G_2) \rightarrow [l]$ οι συναρτήσεις
      χρωματισμού του καθενός.
      
      Επεκτείνουμε τις παραπάνω συναρτήσεις ως εξής:
      \[ \overline{\chi_{G_i}}(u) = \left\{
         \begin{array}{cc}
            \chi_{G_i}(u) & ,\ u \in V(G_i)\\
            1 & , \text{ διαφορετικά}
         \end{array}
         \right.
      \]

      Ορίζουμε το σύνολο $S = \{ (x, y)\ |\ x \in A, y \in B \}$ και
      χρωματίζουμε το $G$ με χρώματα από το $S$ ως εξής:

      \[ \chi_G(u) = \left( \overline{\chi_{G_1}}(u),
                     \overline{\chi_{G_2}}(u) \right) \]

      Ο παραπάνω είναι έγκυρος χρωματισμός αφού αν $\chi_G(u) = \chi_G(v)$
      τότε $\overline{\chi_{G_i}}(u) = \overline{\chi_{G_i}}(v)$
      για $i = 1, 2$ επομένως $\{ u, v \} \notin E(G_i)$ και έτσι
      $\{u, v\} \notin E(G)$.

      Άρα $\chi(G) \leq |S| = \chi(G_1) \cdot \chi(G_2)$.
   \end{proof}

   Από το θεώρημα των 4 χρωμάτων έχουμε ότι αν $G_1, G_2$ επίπεδα γραφήματα
   τότε $\chi(G_1), \chi(G_2) \leq 4$ επομένως από το Λήμμα \ref{lm1.7.1}:
   $\chi(G_1 \cup G_2) \leq 16$.
            
\end{enumerate}

\section{Διαπεράσεις}
\begin{enumerate}
   \item[2.1] \emph{$(\star)$ Για ποιά $k$ και $l$ το γράφημα
              $G_{k, l} = P_l^{[k]}$ είναι Χαμιλτονιανό;}

   Για $k = 1$, κανένα από τα $P_l$ με $l \geq 1$ δεν είναι Χαμιλτονιανό.

   Για $k \geq 2$, θα δείξουμε ότι για κάθε $l \geq 1$ το
   $P_l^{[k]}$ είναι Χαμιλτονιανό.

   \begin{obsrv}
      \label{lm2.1.1}
      Το $P_l^{[2]} = P_l \times P_l$ είναι ισόμορφο με την
      $(l+1, l+1)$-σχάρα η οποία είναι Χαμιλτονιανό γράφημα
      για κάθε $l \geq 1$ (διαπερνάμε όλες τις κορυφές της πρώτης
      στήλης από πάνω προς τα κάτω, της δεύτερης στήλης από κάτω
      προς τα πάνω κ.ο.κ.).
   \end{obsrv}

   \begin{lm}
      \label{lm2.1.2}
      Αν $G$ είναι Χαμιλτονιανό τότε το $G \times P_k$ είναι επίσης
      Χαμιλτονιανό.
   \end{lm}
   \begin{proof}
      Το γράφημα $G \times P_k$ είναι ουσιαστικά το $G$ όπου κάθε κορυφή του
      έχει αντικατασταθεί από ένα μονοπάτι $P_k$ (και έχουν προστεθεί οι
      κατάλληλες ακμές μεταξύ κορυφών των μονοπατιών).

      Ας πάρουμε ένα κύκλο Hamilton του $G$:
      
      \[ u_1 \rightarrow \ldots \rightarrow u_n \rightarrow u_1 \]

      Αυτός μπορεί να μετασχηματιστεί απευθείας
      σε κύκλο Hamilton του $G \times P_k$ ως εξής:
      
      \[ (u_1^1 \rightarrow \ldots \rightarrow u_1^k) \rightarrow
         \ldots \rightarrow (u_n^1 \rightarrow \ldots
         \rightarrow u_n^k) \rightarrow u_1^1 \]

      όπου στο παραπάνω $u_i^j$ είναι η $j$-οστή κορυφή του μονοπατιού
      το οποίο έχει αντικαταστήσει την κορυφή $u_i$ του $G$ στον
      $G \times P_k$.
   \end{proof}

   Από το Λήμμα \ref{lm2.1.2} και την Παρατήρηση \ref{lm2.1.1} έχουμε
   επαγωγικά ότι για κάθε $k \geq 2$ το $P_l^{[k]}$ είναι Χαμιλτονιανό
   για οποιδήποτε $l \geq 1$.
\end{enumerate}

\section{Επίπεδα γραφήματα}
\section{Τέλεια γραφήματα}
\section{Μερικές διατάξεις}
\section{$k$-δέντρα}
\section{Άπειρα γραφήματα}
\section{Κανονικά γραφήματα και Ταιριάσματα}
\section{Διάφορα}

\end{document}
