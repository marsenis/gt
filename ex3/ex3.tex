% --- To be compiled with XeLaTeX ---
% ---      Encoding: UTF-8        ---

\documentclass[a4paper, oneside, 11pt]{article}

%fontspec package provides a configurable interface for font selection, and allows complex font choices to be named and later reused. It's needed for XeLaTeX
\usepackage[cm-default]{fontspec}

% Unicode support
\usepackage{xunicode}
\usepackage{xltxtra}

% Default words and phrases in Greek (e.g. 'Περίληψη' instead of 'Abstract'). Also contains hyphenation rules for Greek Language
\usepackage{xgreek}

% Mathematical fonts, theorems etc.
\usepackage{amsfonts}
\usepackage{amsmath}
\usepackage{amsthm}

% Default page layout for consuming a larger portion of the page.
\usepackage{fullpage}

% Greek fonts (Computer Modern)
\setmainfont[Mapping=tex-text]{CMU Serif}

% Auxiliary commands
\newcommand{\HRule}{\rule{\linewidth}{0.5mm}}

\newtheorem{thm}{Θεώρημα}
\newtheorem{lm}[thm]{Λήμμα}
\newtheorem{obsrv}[thm]{Παρατήρηση}

\theoremstyle{definition}
\newtheorem{defn}[thm]{Ορισμός}

\begin{document}

\begin{titlepage}
\begin{center}

\includegraphics[width=0.3\textwidth]{./pyrforos.png}
\includegraphics[width=0.2\textwidth]{./uoa.png}\\[1cm]

\textsc{\LARGE Σχολή Ηλεκτρολόγων Μηχανικών και Μηχανικών Υπολογιστών}\\[1.5cm]

\HRule \\[0.4cm]
{\huge \bfseries Γραφοθεωρία\\
\LARGE Ομάδα Ασκήσεων No. 2}\\[0.4cm]

\HRule \\[1.5cm]

\begin{center}
\textbf{Ομάδα 7}\\
Αξιώτης Κυριάκος\\
Αρσένης Γεράσιμος
\end{center}

\vfill

{\large \today}
\end{center}

\end{titlepage}


\section{Χρωματισμοί κορυφών και ακμών}
\begin{enumerate}
   \item[1.6] \emph{Έστω $G$ γράφημα όπου $\Delta(G) \leq 3$. Δείξτε ότι το
              $G$ είναι 4-ακμοχρωματίσιμο.}

   Θα δείξουμε ότι γραμμικό γράφημα $L(G)$ του $G$ είναι 4 χρωματίσιμο.

   \begin{lm}
      \label{lm1.6.1}
      Αν $K_4 \subseteq L(G)$ τότε $\Delta(G) \geq 4$.
   \end{lm}
   \begin{proof}
      Έστω $e_1, e_2, e_3, e_4$ οι ακμές του $G$ που στο $L(G)$ είναι
      κορυφές 4-κλίκας. Αυτό σημαίνει ότι κάθε ζεύγος $e_i, e_j$ θα πρέπει
      να έχει κοινό άκρο.

      Έστω $e_1 = \{u, v\}$ και χωρίς βλάβη της γενικότητας έστω
      $e_2 = \{u, w\}$. Αν η $e_3$ έχει κοινό άκρο με την $e_1$
      την κορυφή $v$, τότε αναγκαστικά $e_3 = \{v, w\}$ ώστε να έχει
      κοινό άκρο και με την $e_3$. Σε αυτή την περίπτωση όμως η $e_4$
      δεν μπορεί να έχει κοινό άκρο και με τις 3 προηγούμενες ακμές.

      Άρα η $e_3$ έχει κοινό άκρο με την $e_1$ το $u$, δηλαδή
      $e_3 = \{ u, x \}$ για κάποια κορυφή $x$ (διαφορετική από
      τις $\{u, v, w\}$).

      Τέλος, η $e_4$ θα πρέπει να έχει κοινό άκρο με όλες τις υπόλοιπες
      και αυτό μπορεί να συμβεί μόνο αν $e_4 = \{u, y\}$ για κάποια
      νέα κορυφή $y$.

      Συνεπώς $\Delta(G) \geq d(u) = 4$.
   \end{proof}

   Εφόσον $\Delta(G) \leq 3$, από το Λήμμα \ref{lm1.6.1} έχουμε ότι
   το $L(G)$ δεν μπορεί να περιέχει το $K_4$ ως υπογράφημα άρα δεν μπορεί
   να το περιέχει και ως ελάσσον.

   Από την εικασία του Hadwinger για την περίπτωση $k=4$ (για το συγκεκριμένο
   $k$ έχει αποδειχθεί ότι η εικασία ισχύει) έχουμε ότι $\chi(L(G)) < 4$
   άρα μπορούμε να χρωματίσουμε τις ακμές του $G$ με 4 (ή λιγότερα)
   χρώματα.

   % TODO: Check the proof again. Edeiksa kati pio isxyro opote mporei
   %       na exw kanei lathos.

   \item[1.7] \emph{Δείξτε ότι υπάρχει $c$ τέτοιο ώστε κάθε ένωση δύο επίπεδων
              γραφημάτων να έχει χρωματικό αριθμό το πολύ $c$.}
   
   \begin{lm}
      \label{lm1.7.1}
      Αν $G = G_1 \cup G_2$ τότε $\chi(G) \leq \chi(G_1) \cdot \chi(G_2)$.
   \end{lm}
   \begin{proof}
      Έστω $\chi(G_1) = k, \chi(G_2) = l$
      και $\chi_{G_1} : V(G_1) \rightarrow [k],
      \chi_{G_2} : V(G_2) \rightarrow [l]$ οι συναρτήσεις
      χρωματισμού του καθενός.
      
      Επεκτείνουμε τις παραπάνω συναρτήσεις ως εξής:
      \[ \overline{\chi_{G_i}}(u) = \left\{
         \begin{array}{cc}
            \chi_{G_i}(u) & ,\ u \in V(G_i)\\
            1 & , \text{ διαφορετικά}
         \end{array}
         \right.
      \]

      Ορίζουμε το σύνολο $S = \{ (x, y)\ |\ x \in A, y \in B \}$ και
      χρωματίζουμε το $G$ με χρώματα από το $S$ ως εξής:

      \[ \chi_G(u) = \left( \overline{\chi_{G_1}}(u),
                     \overline{\chi_{G_2}}(u) \right) \]

      Ο παραπάνω είναι έγκυρος χρωματισμός αφού αν $\chi_G(u) = \chi_G(v)$
      τότε $\overline{\chi_{G_i}}(u) = \overline{\chi_{G_i}}(v)$
      για $i = 1, 2$ επομένως $\{ u, v \} \notin E(G_i)$ και έτσι
      $\{u, v\} \notin E(G)$.

      Άρα $\chi(G) \leq |S| = \chi(G_1) \cdot \chi(G_2)$.
   \end{proof}

   Από το θεώρημα των 4 χρωμάτων έχουμε ότι αν $G_1, G_2$ επίπεδα γραφήματα
   τότε $\chi(G_1), \chi(G_2) \leq 4$ επομένως από το Λήμμα \ref{lm1.7.1}:
   $\chi(G_1 \cup G_2) \leq 16$.
\end{enumerate}

\section{Διαπεράσεις}
\begin{enumerate}
   \item[2.1] \emph{$(\star)$ Για ποιά $k$ και $l$ το γράφημα
              $G_{k, l} = P_l^{[k]}$ είναι Χαμιλτονιανό;}

   Για $k = 1$, κανένα από τα $P_l$ με $l \geq 1$ δεν είναι Χαμιλτονιανό.

   Για $k \geq 2$, θα δείξουμε ότι για κάθε $l \geq 1$ το
   $P_l^{[k]}$ είναι Χαμιλτονιανό.

   \begin{obsrv}
      \label{lm2.1.1}
      Το $P_l^{[2]} = P_l \times P_l$ είναι ισόμορφο με την
      $(l+1, l+1)$-σχάρα η οποία είναι Χαμιλτονιανό γράφημα
      για κάθε $l \geq 1$ (διαπερνάμε όλες τις κορυφές της πρώτης
      στήλης από πάνω προς τα κάτω, της δεύτερης στήλης από κάτω
      προς τα πάνω κ.ο.κ.).
   \end{obsrv}

   \begin{lm}
      \label{lm2.1.2}
      Αν $G$ είναι Χαμιλτονιανό τότε το $G \times P_k$ είναι επίσης
      Χαμιλτονιανό.
   \end{lm}
   \begin{proof}
      Το γράφημα $G \times P_k$ είναι ουσιαστικά το $G$ όπου κάθε κορυφή του
      έχει αντικατασταθεί από ένα μονοπάτι $P_k$ (και έχουν προστεθεί οι
      κατάλληλες ακμές μεταξύ κορυφών των μονοπατιών).

      Ας πάρουμε ένα κύκλο Hamilton του $G$:
      
      \[ u_1 \rightarrow \ldots \rightarrow u_n \rightarrow u_1 \]

      Αυτός μπορεί να μετασχηματιστεί απευθείας
      σε κύκλο Hamilton του $G \times P_k$ ως εξής:
      
      \[ (u_1^1 \rightarrow \ldots \rightarrow u_1^k) \rightarrow
         \ldots \rightarrow (u_n^1 \rightarrow \ldots
         \rightarrow u_n^k) \rightarrow u_1^1 \]

      όπου στο παραπάνω $u_i^j$ είναι η $j$-οστή κορυφή του μονοπατιού
      το οποίο έχει αντικαταστήσει την κορυφή $u_i$ του $G$ στον
      $G \times P_k$.
   \end{proof}

   Από το Λήμμα \ref{lm2.1.2} και την Παρατήρηση \ref{lm2.1.1} έχουμε
   επαγωγικά ότι για κάθε $k \geq 2$ το $P_l^{[k]}$ είναι Χαμιλτονιανό
   για οποιδήποτε $l \geq 1$.

   \item[2.11] \emph{$(\star)$ Ένα τριγωνοποιημένο επίπεδο γράφημα έχει
               χρωματικό αριθμό 3 αν και μόνο αν είναι γράφημα Euler.}

   Θα θεωρήσουμε ότι το γράφημα περιέχει τουλάχιστον 3 κορυφές αφού
   διαφορετικά η πρόταση είναι τετριμμένη.

   Δείχνουμε τις δύο κατευθύνσεις της εκφώνησης ως εξής:

   \begin{itemize}
      \item[$(\Rightarrow)$]
         Έστω (προς απαγωγή σε άτοπο) ότι το $G$ (με $n(G) \geq 3$)
         τριγωνοποιημένο επίπεδο γράφημα το οποίο είναι 3-χρωματίσιμο
         αλλά \emph{δεν} είναι γράφημα Euler.

         Το $G$ θα πρέπει να περιέχει τουλάχιστον μία κορυφή περιττού
         βαθμού, έστω $u \in V(G)$. Η $u$ δεν μπορεί να έχει βαθμό 1
         γιατί διαφορετικά θα βρίσκεται στο σύνορο μίας μόνο
         όψης $f$ η οποία όμως θα πρέπει να έχει στο σύνορό της τουλάχιστον
         άλλες 2 κορυφές. Έστω $v, w$ αυτές οι κορυφές και χωρίς βλάβη της
         γενικότητας έστω $v$ η γειτονική της $u$. Τότε όμως μπορούμε
         να προσθέσουμε την ακμή $\{w, u\}$ και το γράφημα να παραμείνει
         επίπεδο. Αυτό είναι άτοπο γιατί το γράφημα είναι τριγωνοποιημένο,
         δηλαδή η προσθήκη μιας ακμής δεν θα έπρεπε να είναι εφικτή.

         Συνεπώς $d(u) \geq 3$. Έστω $[v_0, v_1, \ldots, v_{k-1}]$
         οι γειτονικές
         κορυφές τις $u$ σε ορολογιακή διάταξη όπως εμφανίζονται στην
         επίπεδη εμβάπτιση του $G$. Αφού το γράφημα είναι τριγωνοποιημένο
         θα πρέπει να υπάρχουν οι ακμές $\{v_i, v_{(i+1) \mod k}\}$ για κάθε
         $i = 0, \ldots, k-1$.

         Άρα η γειτονιά της $u$ ενάγει περιττό κύκλο και αυτό σημαίνει
         ότι χρειάζονται τουλάχιστον 4 χρώματα για το χρωματισμό
         της $u$ και της γειτονιάς της. Άτοπο.

      \item[$(\Leftarrow)$]
         Έστω τριγωνοποιημένο επίπεδο γράφημα $G$ με $n(G) \geq 3$
         το οποίο είναι γράφημα Euler αλλά \emph{δεν} είναι
         3-χρωματίσιμο.

         Από την εικασία του Hadwinger για $k=4$, έχουμε ότι
         $K_4 \leq G$, δηλαδή υπάρχει μια ακολουθία συνθλίψεων
         ακμών μετά από την οποία το γράφημα $G'$ που απομένει
         περιέχει 4-κλίκα.

         Κάθε κορυφή του $G$ έχει άρτιο βαθμό (ως γράφημα Euler) και
         έτσι το ίδιο θα ισχύει και για κάθε γράφημα που προκύπτει
         από συνθλίψεις ακμών του $G$. Συνεπώς το $G'$ θα είναι
         γράφημα Euler.

         Έστω $x, y, z, w$ οι κορυφές τις 4-κλίκας του $G'$.

         TODO: ... test
   \end{itemize}
\end{enumerate}

\section{Επίπεδα γραφήματα}
\section{Τέλεια γραφήματα}
\section{Μερικές διατάξεις}
\section{$k$-δέντρα}

\begin{enumerate}
   \item[6.2] \emph{Καλούμε μερικό $k$-δέντρο κάθε υπογράφημα $k$-δέντρου.
   Δείξτε ότι το $K_{r,r}$ είναι μερικό $r$-δέντρο αλλά δεν είναι μερικό
   $(k-1)$-δέντρο.}

   Το $K_{r,r}$ είναι μερικό $k$-δέντρο αφού μπορούμε να το παράγουμε
   ως εξής:

   Ξεκινάμε με το $K_{r+1}$ και διαλέγουμε μία κορυφή του την οποία
   αναθέτουμε στο σύνολο $X$ και τις υπόλοιπες τις αναθέτουμε στο σύνολο
   $Y$. Το $Y$ είναι μια $r$-κλίκα επομένως μπορούμε να τοποθετήσουμε
   $r-1$ νέες κορυφές στο $X$ κάθε μία από τις οποίες τις συνδέουμε
   με όλες τις κορυφές του $Y$.

   Τώρα αφαιρούμε όλες τις ακμές μεταξύ κορυφών του $Y$ και αυτό
   που μένει είναι το $K_{r,r}$.

   Έστω τώρα ότι το $K_{r,r}$ ήταν μερικό $(r-1)$-δέντρο. Τότε θα πρέπει
   να περιέχει μια κορυφή $u$ με $d(u) < r$ (η τελευταία κορυφή που προσθέσαμε
   κατα της κατασκευή του $(r-1)$-δέντρου είχε βαθμό $r-1$). Αυτό όμως είναι
   άτοπο γιατί όλες οι κορυφές του $K_{r,r}$ έχουν βαθμό ίσο με $r$.
\end{enumerate}

\section{Άπειρα γραφήματα}

\begin{enumerate}
   \item[7.3] \emph{$(\star)$ Χρησιμοποιώντας το λήμμα του K\H{o}nig, αποδείξτε
   ότι αν το $G$ είναι γράφημα όπου $|V(G)| = \aleph_0$ και κάθε υπογράφημά
   του είναι 3-χρωματίσιμο, τότε και το $G$ είναι 3-χρωματίσιμο.}

   Έστω $V(G) = \{ 1, 2, \ldots, n, \ldots \}$. Συμβολίζουμε με $G[k]$
   το εναγόμεμο υπογράφημα του $G$ με κορυφές τις $\{1, \ldots, k\}$.

   Δημιουργούμε το εξής δέντρο $T$: Κάθε κόμβος του δέντρου εκτός της ρίζας
   αντιστοιχεί σε ένα έγκυρο 3-χρωματισμό του $G[k]$ για κάποιο $k$.
   Συγκεκριμένα, η ρίζα έχει 3 παιδιά που αντιστοιχούν στους τρεις πιθανούς
   χρωματισμούς του $G[1]$ και αν ένας κόμβος $u \in T$ αντιστοιχεί σε
   3-χρωματισμό
   του $G[k]$, τότε θεωρούμε το γράφημα $G[k+1] \supseteq G[k]$ καθώς και κάθε
   3-χρωματισμό του που συμφωνεί με το χρωματισμό του $G[k]$. Υπάρχουν
   3 τέτοιοι χρωματισμοί (3 επιλογές για το χρώμα της νεας κορυφής).
   και ώς παιδία της $u$ θέτουμε τους έγκυρους από αυτούς
   τους χρωματισμούς.

   Παρατηρούμε ότι ένας κόμβος $u$ βρίσκεται σε απόσταση $r$ από τη ρίζα
   του $T$ αν και μόνο αν το $u$ αντιστοιχεί σε έγκυρο 3-χρωματισμό
   του $G[r]$.

   Για το γράφημα $T$ γνωρίζουμε ότι κάθε κόμβος έχει πεπερασμένο βαθμό
   (το πολύ 4) και ότι έχει άπειρο πλήθος κόμβων γιατί σύμφωνα με την
   προηγούμενη παρατήρηση, αν τo $G[k]$ είναι 3-χρωματίσιμο θα πρέπει
   να υπάρχει τουλάχιστον μια κορυφή $u$ που να αντιστοιχεί στο
   χρωματισμό του. Ξέρουμε όμως ότι όλα τα $G[k]$ για $k \in \mathbb{N}$
   είναι 3-χρωματίσιμα άρα θα πρέπει να υπάρχει τουλάχιστον μια κορυφή
   για κάθε τέτοιο $k$.

   Από το λήμμα του K\H{o}nig έχουμε λοιπόν ότι πρέπει να υπάρχει άπειρο
   μονοπάτι $P$ που να ξεκινάει από τη ρίζα. Το μονοπάτι αυτό ορίζει
   έναν 3-χρωματισμό του $G$ (το χρώμα μιας κορυφής $w \in V(G)$
   είναι το χρώμα που του αναθέτει ο χρωματισμός του $G[w]$ στο μονοπάτι
   $P$). O χρωματισμός αυτός είναι έγκυρος γιατί διαφορετικά, αν υπάρχουν
   κορυφές $u, v \in V(G)$ με $\{u, v\} \in E(G)$ και ίδιο χρώμα, τότε
   ο χρωματισμός του $G[\max(u, v)]$ στο μονοπάτι $P$ δεν θα ήταν έγκυρος.
\end{enumerate}

\section{Κανονικά γραφήματα και Ταιριάσματα}
\section{Διάφορα}

\begin{enumerate}
   \item[9.7] \emph{$(\star)$ Ποιά είναι η συνεκτικότητα του υπερκύβου
   $r$ διαστάσεων;}

   Θα δείξουμε με επαγωγή ότι $\kappa(Q_r) = r$.

   Για $r = 1$ το $Q_1$ περιέχει μόνο μία ακμή και είναι συνεκτικό.

   Αν ο $Q_{r-1}$ είναι $(r-1)$-συνεκτικός τότε θα δείξουμε ότι
   ο $Q_r = Q_{r-1} \times P_1$ είναι $r$-συνεκτικός.

   Ο $Q_r$ ως γνωστόν αποτελείται από δύο αντίγραφα $A_1, A_2$ του $Q_{r-1}$
   μαζί με τις ακμές που συνδέουν αντίστοιχες κορυφές μεταξύ τους.
   Στο εξής, αν έχουμε μια κορυφή $u \in V(A_1)$ θα συμβολίζουμε
   με $u'$ την κορυφή του $A_2$ με την οποία συνδέεται η $u$ στο
   $Q_r$.

   Θα δείξουμε ότι για οποιεσδήποτε δύο κορυφές $u, v \in V(Q_r)$
   υπάρχουν $r$ εσωτερικώς διακεκριμένα μονοπάτια από την $u$ στην
   $v$ διακρίνοντας τις εξής περιπτώσεις:

   \begin{itemize}
      \item $u, v \in V(A_1)$ (αντίστοιχα και για το $A_2$).

      Από την Ε.Υ. υπάρχουν $r-1$ εσωτερικώς διακεκριμένα μονοπάτια
      από την $u$ στην $v$ που χρησιμοποιούν μόνο ακμές μόνο από το
      $A_1$. Επίσης υπάρχει τουλάχιστον ένα μονοπάτι $P$ μεταξύ των
      $u'$ και $v'$ στο $A_2$ επομένως μπορούμε να δημιουργήσουμε
      το $P' = [u, u'] \cup P \cup [v', v]$ που δεν έχει κοινές κορυφές με
      τα υπόλοιπα $r-1$ εκτός από τα άκρα.

      \item $u \in V(A_1)$ και $v \in V(A_2)$ (ή αντίστροφα).

      Έστω $P_i$ για $i = 1, \ldots, r-1$ τα $r-1$ εσωτερικώς
      διακεκριμένα μονοπάτια μεταξύ των $u$ και $v$ στο $A_1$ και
      $P_i'$ τα αντίστοιχα μονοπάτια στο $A_2$. Συβολίζουμε με
      $x_i$ τον προτελευταίο κόμβο του μονοπατιού $P_i$.

      Με βάση τα μονοπάτια αυτά δημιουργούμε τα παρακάτω $r$
      εσωτερικώς διακεκριμένα μονοπάτια $R_i$:

      \begin{equation*}
         R_i = \left\{
            \begin{array}{cl}
               [u, u']\cup P_1' &, \ i = 1\\
               (P_i \backslash v) \cup [x_i, x_i', v'] &, \ i = 2, \ldots, r-1\\
               P_1 \cup [v, v'] &,\ i = r\\
            \end{array}
            \right.
      \end{equation*}

      %TODO: Να βάλουμε σχήμα
   \end{itemize}
\end{enumerate}

\end{document}
