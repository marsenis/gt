% --- To be compiled with XeLaTeX ---
% ---      Encoding: UTF-8        ---

\documentclass[a4paper, oneside, 11pt]{article}

%fontspec package provides a configurable interface for font selection, and allows complex font choices to be named and later reused. It's needed for XeLaTeX
\usepackage[cm-default]{fontspec}

% Unicode support
\usepackage{xunicode}
\usepackage{xltxtra}

% Default words and phrases in Greek (e.g. 'Περίληψη' instead of 'Abstract'). Also contains hyphenation rules for Greek Language
\usepackage{xgreek}

% Mathematical fonts, theorems etc.
\usepackage{amsfonts}
\usepackage{amsmath}
\usepackage{amsthm}

% Default page layout for consuming a larger portion of the page.
\usepackage{fullpage}

% Greek fonts (Computer Modern)
\setmainfont[Mapping=tex-text]{CMU Serif}

% Auxiliary commands
\newcommand{\HRule}{\rule{\linewidth}{0.5mm}}

\newtheorem{thm}{Θεώρημα}
\newtheorem{lm}[thm]{Λήμμα}

\theoremstyle{definition}
\newtheorem{defn}[thm]{Ορισμός}

\begin{document}

\begin{titlepage}
\begin{center}

\includegraphics[width=0.3\textwidth]{./pyrforos.png}
\includegraphics[width=0.2\textwidth]{./uoa.png}\\[1cm]

\textsc{\LARGE Σχολή Ηλεκτρολόγων Μηχανικών και Μηχανικών Υπολογιστών}\\[1.5cm]

\HRule \\[0.4cm]
{\huge \bfseries Γραφοθεωρία\\
\LARGE Ομάδα Ασκήσεων No. 2}\\[0.4cm]

\HRule \\[1.5cm]

\begin{center}
\textbf{Ομάδα 7}\\
Αξιώτης Κυριάκος\\
Αρσένης Γεράσιμος
\end{center}

\vfill

{\large \today}
\end{center}

\end{titlepage}


\begin{enumerate}

\item[1.]
   Σε ένα $G(n, p)$ η πιθανότητα μιας κορυφής να έχει βαθμό $k$ είναι
   ${n-1 \choose k} p^k (1-p)^{n-1-k}$. Δείξτε ότι ο μέσος βαθμός είναι
   $(n-1)p$ με απευθείας υπολογισμό, δηλαδή χωρίς να χρησιμοποιήσετε τη
   γραμμικότητα της μέσης τιμής.

   \begin{proof}
   \end{proof}

\item[2.]
   Δείξτε ότι το τυχαίο γράφημα $G(n, p)$ με $p = n^{-0.7}$ δεν έχει σχεδόν
   σίγουρα 4-κλίκα για αρκετά μεγάλα $n$.

   \begin{proof}
   \end{proof}

\item[3. ($\star$)]
   Θεωρήστε το παρακάτω τυχαίο κατευθονόμενο γράφημα. Για κάθε κορυφή $v$
   επιλέγουμε ομοιόμορφα τυχαία μια κορυφή $u$ και τοποθετούμε την ακμή
   $v \rightarrow u$. Κάθε κορυφή έχει μόνο μια εξερχόμενη ακμή και μπορεί να
   υπάρχουν θηλιές. Έστω $r(v)$ ο αριθμός των κορυφών στις οποίες μπορούμε να
   φτάσουμε από την $v$.

   \begin{itemize}
      \item Για $k = 1, \ldots, n$ ποιά η πιθανότητα $r(v) = k$. Η πιθανότητα θα
            έχει μορφή γινομένου.
      \item Δείξτε ότι για μία κορυφή $v$, $Pr[ r(v) \leq \sqrt{n} / 10 ]
            \leq 1/3$ και $Pr [ r(v) \geq 10\sqrt{n} ] \leq 1/3$.
   \end{itemize}

   \begin{proof}
   \end{proof}

\item[4. ($\star$)]
   Θεωρήστε το τυχαίο γράφημα $G(n, p)$ με $p = 6.6/n$. Δείξτε ότι το γράφημα
   είναι σχεδόν σίγουρα μή 3-χρωματίσιμο για αρκετά μεγάλα $n$.

   \begin{proof}
   \end{proof}

\item[5. ($\star$)]
   Θεωρήστε το παρακάτω τυχαίο γράφημα με $n$ κορυφές. Κάθε κορυφή διαλέγει
   ομοιόμορφα τυχαία 2 κορυφές και τοποθετούμε μη-κατευθυνόμενες ακμές προς
   αυτές. Η τυχαία επιλογή γίνεται με επανάληψη και μπορεί μια κορυφή $v$ να
   επιλέξει και τον εαυτό της στην οποία περίπτωση παραλείπουμε αυτή τη θηλιά.
   Παρατηρούμε ότι οι ακμές θα είναι περίπου $2n$ αλλά μπορεί κάποιες κορυφές να
   έχουν βαθμό μικρότερο από $2$ αν επέλεξαν τον εαυτό τους ή την ίδια κορυφή
   δύο φορές. Μπορεί επίσης κάποιες κορυφές να έχουν βαθμό αρκετά μεγαλύτερο από
   $4$ αν άλλες κορυφές έτυχε να τις επιλέξουν.

   Δείξτε ότι το γράφημα είναι σχεδόν σίγουρα συνεκτικό για αρκετά μεγάλα $n$.

   \begin{proof}
   \end{proof}

\end{enumerate}

\end{document}
